\section{Exact methods}

Problem \ref{prob:min} can be formulated as an integer linear program and solved to optimality. In this section, we present two different modeling approaches. 

\subsection{Broadcast time model}
The first studied model is a straightforward formulation of the problem.
Consider variables 
$$ x_{ij}^k=
\begin{cases} 
1, \text{ if } j\in V_k \text{ and } \pi(j)=i,\\ 
0, \text{ otherwise},
\end{cases}
$$
and a variable $c$ representing the number of necessary time steps.
The worst case scenario is when $G$ is a path $v_1,\dots,v_n$ with $S=\{v_1\}$. 
In such an instance, the necessary number of time steps is $n-1$, which gives a trivial upper bound on $c$.
Problem \ref{prob:min} is then formulated as follows: 

\begin{subequations}\label{mod:basic}
\begin{align}
\label{mod:basic:obj} \min c \\ 
\label{mod:basic:onefromroot} \sum_{i \in N(s)}x^1_{si} & \leq 1 & s\in S,\\
\label{mod:basic:singlein} \sum\limits_{k=1}^{n-1}\sum\limits_{j\in N(i)}x_{ji}^k & = 1 & i\in V \setminus S,\\
\label{mod:basic:uniqueTout} \sum\limits_{j\in N(i)}x_{ij}^k & \leq 1  & k=1,\dots,n-1,i\in V,\\
\label{mod:basic:tIncreases} \sum\limits_{j\in N(i)}x_{ij}^k &\leq\sum\limits_{\ell=1}^{k-1}\sum\limits_{v\in N(i)\setminus\{j\}} x_{vi}^{\ell}  & i\in V\setminus S, k=2,\dots,n-1,\\
\label{mod:basic:tcrel} \sum\limits_{k=1}^{n-1}k\cdot x_{ij}^k & \leq c &  (i,j)\in A,\\
%\label{mod:basic:tcrel} \sum\limits_{t=1}^{n-1}t\sum\limits_{j\in N(i)}x_{ij}^k & \leq c &  i\in V,\\
\label{mod:basic:positiveCost}x_{ij}^1 & = 0 & (i,j)\in A, i \not\in S,\\
\label{mod:basic:dim}&&x \in \{0,1\}^{A\times V}, c \in\{1,\dots,n-1\}.
\end{align}~
\end{subequations}

Constraints \eqref{mod:basic:onefromroot} indicate that for each source node $s$, there is at most one adjacent node $i\in V_1$ such that $\pi(i)=s$.
By \eqref{mod:basic:singlein}, for every non-source node, there is exactly one node $j$ such that $\pi(i)=j$.
Constraints \eqref{mod:basic:uniqueTout} enforce that for each node $i\in V$ and each subset $V_k$, there is at most one adjacent node $j\in V_k$ with $\pi(j)=i$.
The requirement that a non-source node has a neighbor $j\in V_k:\pi(j)=i$ only if there exists $v\in V_{k-1}: \pi(i)=v$ is modeled by \eqref{mod:basic:tIncreases}. 
%The requirement that only informed nodes can relay a signal is modeled by \eqref{mod:basic:tIncreases}. 
%The maximum time step at which any transmission takes place is captured by \eqref{mod:basic:tcrel}, and finally, \eqref{mod:basic:positiveCost} states that a node that is not a source never transmits in the first time step.
The length of the sequence of subsets is captured by \eqref{mod:basic:tcrel}, and finally, \eqref{mod:basic:positiveCost} state that if $\pi(j)\not\in S$ for some $j\in V$, then $j\not\in V_1$.
\subsection{Binomial tree model}

The following method solves Problem \ref{prob:min} by solving a sequence of decision problems:
\begin{problem}
\label{prob:dec}
Given $G=(V,E)$, $S\subseteq V$ and $t\in \mathbb{N}$, is there a sequence $S=V_0\subseteq\dots\subseteq V_t=V$ 
and a mapping $\pi:V\setminus S\to V$, such that for each $v\in V\setminus S:\{v,\pi(v)\}\in E$ and for each  $u,v\in V_i\setminus V_{i-1}: \pi(u)\neq \pi(v)\Leftrightarrow u\neq v$?
\end{problem}
For a delay $t$, at most $s\cdot 2^t$ nodes can be informed within $t$ steps. 
%Therefore, we assume $n\leq 2^ks.
This can be achieved when the broadcast forest $T$ consists of binomial trees $B^t$ of order $t$ rooted at sources $s\in S$.
Hence, if there is a partition of $G$ into $s$ pruned binomial trees of order at most $t$ rooted at sources, then $(G,S,t)$ is a YES instance of Problem \ref{prob:dec}.
%Finding a partition of $G$ into $s$ pruned binomial trees can be equivalently formulated as finding a partition of $G''$ into $s$ (complete) binomial trees, 
%where $G''$ is constructed from $G$ as follows:
%Let $\alpha\coloneqq s\cdot 2^k-|V|$, and let $K_\alpha=(V_\alpha,E_\alpha)$ be a complete graph on $\alpha$ nodes.
%Each node in $K_\alpha$ is connected to every node $v\in V$ in the original graph $G$.
%Thus, $G''=(V'',E'')$ with $V''=V\cup V_\alpha$ and $E''=E\cup E_\alpha\cup \{\{u,v\}: u\in V \wedge v\in V_\alpha\}$.
%The set of arcs $A''$ is constructed by creating two arcs of opposite orientation for each edge, but arcs with orientation from $V_\alpha$ to $V$ are excluded. 
%Formally, $A''=A\cup\{(u,v),(v,u): \{u,v\}\in E_\alpha\}\cup\{(u,v):u\in V \wedge v\in V_\alpha\}$.

Let $I=\{1,\dots,2^t\}$.
For a directed rooted binomial tree $B^t=(V^t,A^t)$, we define a systematic numbering of nodes in $V^t$, so that a node number determines a unique position in $B^t$.
I.e., we need a bijective mapping $\beta: V^t \to I$.
A suitable mapping $\beta$ assigns values increasingly with decreasing outgoing degree. 
If there is an ambiguity, a node whose parent has a lower number is assigned a lower number.
This mapping is defined recursively as
\begin{equation}
\label{eq:beta}
\beta(v)=\begin{cases}
1,\text{ if } v\in S \text{ is a root of } B^t,\\
\beta(\pi(v)) + 2^{t-deg^+(v)-1}, \text{ otherwise}.
\end{cases}
\end{equation}
\begin{observation}\label{obs:deg}
For each $i\in\{1,\dots,t\}$, the set $\{v\in V^t: 1\leq\beta(v)\leq2^i\}\setminus\{v\in V^t:1\leq\beta(v)\leq2^{i-1}\}$ contains nodes with out-degree $t-i$.
\end{observation}
\begin{observation}\label{obs:childdeg}
Children of $v\in V^t$ with $\text{deg}^+(v)=\ell$ have out-degree $0,\dots,\ell-1$.
\end{observation}
\begin{proposition}\label{lem:probeq}
An instance $(G,S,t)$ of Problem \ref{prob:dec} is a YES-instance iff 
there exists a partition of $G$ into $s$ directed pruned binomial trees $B^t_1,\dots,B^t_{s}$ of order $t$ rooted at sources in $S$.
%such that for each $B^t_i=(V^t_i,A^t_i)$ we have that $A^t_i\cap\{(u,v):u \in V_\alpha \wedge v\in V^t_i\}=\emptyset$. 
\end{proposition}
\begin{proof}
Assume there is a node partition of $G$ into directed rooted binomial trees $B^t_1,\dots B^t_{s}$. 
Nodes in $V$ are divided into a sequence of subsets $S=V_0\subseteq\dots\subseteq V_t=V$ such that $V_i=\{v\in V:1\leq\beta(v)\leq 2^i\}$,
and $\pi(v)$ is a parent of $v$ in a corresponding binomial tree.
For $u,v\in V_i\setminus V_{i-1}$ we have from Observation \ref{obs:deg} that $deg^+(u)=deg^+(v)=t-i$.
Furthermore, $\pi(u)\neq \pi(v)$ must hold, because due to \ref{obs:childdeg} no node has two children with the same out-degree.
These observations can be applied because the mapping $\pi$ corresponds to arcs in binomial trees $B^t_1,\dots,B^t_{s}$. 

Conversely, suppose there is a sequence of subsets and a mapping $\pi$ in $G$ with the desired properties.
For $1\leq i\leq t$, $|V_i|\geq2|V_{i-1}|$ must hold, because otherwise $\forall u,v\in V_i:\pi(u)\neq \pi(v)\Leftrightarrow u\neq v$ could not be satisfied.
For some node $v\in V_i$, there is at most $t-i$ nodes $u$ such that $\pi(u)=v$.
Pruned binomial trees covering $G$ can then be constructed simply by following the mapping $\pi$.
%The remaining arcs that complement binomial trees are distributed along $A''\setminus A$.
\qed
\end{proof}

Consider a graph $G'=(V',E')$ constructed  by adding a universal node $v_0$ to $G$. 
The set of nodes and edges is then $V'=V\cup \{v_0\}$ and $E'=E\cup\{\{v_0,v\}:v\in V\}$.
The ILP model based on partition into binomial trees uses only one type of variables
$$
y_{is}^v=\begin{cases}
1, \text{ if  node } v \text{ is the } i\text{-th node of the binomial tree rooted at } s\in S,\\
0, \text{ otherwise},
\end{cases}
$$
where $v\in V'$ and $i\in I $. 
With the definition of $G'$ above, it is straightforward to specify constraints that enforce desired values for $y$-variables.
Whenever $y_{is}^{v_0}=1$, it indicates that the binomial tree $B^t_s$ was pruned at node $i$.
%An obvious weakness of this approach is  that the number of nodes increases to $|V''|=\mathcal{O}(ns)$, and the dimension of variables is thus $\mathcal{O}(n^2s^2)$.
%However, once a suitable partition is found, the arcs of binomial trees contained in $K_\alpha$ can be diversely shuffled while preserving the layour of binomial trees in $G$.
%Instead of adding the entire complete graph $K_\alpha$, a single node $v_0$ with a loop $(v_0,v_0)$ is connected as an apex to the original $G$.
%Let us denote this multigraph as $G'=(V',E')$, where $V'=V\cup\{v_0\}$, $E'=E\cup\{\{u,v_0\}:u\in V\}\cup\{\{v_0\}\}$. 
%The arc set is then analogously defined as $A'=A\cup\{(u,v_0): u\in V\}\cup\{(v_0,v_0)\}$.
%The requirement for partition into binomial trees has to be adjusted accordingly.
%The subtrees contained in $G$ remain unchanged, every arc $(u,v)\in A^k_i, i=1,\dots,s$ in $G''$ with $u\in V$ and $v\in V_\alpha$ becomes $(u,v_0)$ in $G'$,
%and every $(u,v)\in A_\alpha$ becomes $(v_0,v_0)$.
%So, $v_0$ acts as a universal node that can substitute several nodes in each binomial tree.

Let us define the set $C^1(i)$ of $\beta$-values of children (direct descendants) of node $v$ with $\beta(v)=i$ in $B^t$:
\begin{equation}
C^1(i)=\{2^j+i:j=\lceil\log_2 i\rceil,\dots,t-1\}.
\end{equation}
This definition can be generalized to descendants of an arbitrary distance $k$ from $v$ in $B^t$:
\begin{equation}
C^{k+1}(i)=\{C^k(j):j\in C^1(i)\}.
\end{equation}

The values of the variables must fulfill constraints

\begin{subequations}\label{mod:partition}
\begin{align}
\label{mod:part:nodeBelongs} \sum\limits_{i\in I}\sum\limits_{s\in S}y^v_{is} & = 1 & v\in V,\\
\label{mod:part:treeHasIJ} \sum\limits_{v\in V'}y^v_{is} & = 1 & i\in I,s\in S,\\
\label{mod:part:source1} y_{1s}^s & = 1  & s\in S,\\
%\label{mod:part:noReturn} y^u_{ij}+y^v_{lj} &\leq 1 & i\in I,l\in C(i), j\in J, u\in V_\alpha,v\in V,\\
%\label{mod:part:followArcs} y^u_{is}+y^v_{\ell s} &\leq 1 & i\in I,\ell\in C(i), s\in S, u,v\in V',(u,v)\not\in A',\\
\label{mod:part:followArcsA} y^u_{is}+y^v_{\ell s} &\leq 1 & i\in I,\ell\in C(i), s\in S, u,v\in V,\{u,v\}\not\in E,\\
\label{mod:part:followArcsB} y^{v_0}_{is}+y^v_{\ell s} &\leq 1 & i\in I,\ell\in C(i), s\in S, v\in V,\\
\label{mod:part:dim}&&y \in \{0,1\}^{I\times S\times V'}.
\end{align}~
\end{subequations}

As the model represents the decision problem, it suffices to find any feasible solution, and no objective function is needed.
The interpretation of constraints \eqref{mod:part:nodeBelongs} is that every node in the original graph $G$ belongs to exactly one binomial tree.
Note that these constraints are quantified only over $V$ and not over $V'$.
In this way it is achieved that $v_0$ can be regarded as a part of several binomial trees.
By \eqref{mod:part:treeHasIJ} is ensured that there is always exactly one $i$-th node of each binomial tree.
By the summation over $V'$ is ensured, that pruned nodes are collectively represented by $v_0$.
Next, \eqref{mod:part:source1} enforce that source nodes are always the first nodes in corresponding binomial trees, in accordance with definition \eqref{eq:beta} of the mapping $\beta$.
The remaining two sets of constraints guarantee that the arcs of binomial trees follow edges in $E$.
In particular, it is enforced by \eqref{mod:part:followArcsA} that if $u$ and $v$ are not adjacent in $G$, then $v$ must not act as a child of $u$ in any binomial tree.
Finally, \eqref{mod:part:followArcsB} forbids any node from $V$ to be a child of $v_0$ in any binomial tree. 
This reflects the obvious fact that if a tree is pruned at some node, all its descendants must also be excluded from the tree.
%The definition of $A'$ also prevents arcs of the binomial trees to be oriented from $V_\alpha$ to $V$.
%In other words, once the signal leaves the original graph $G$ and enters $v_0$, it cannot return back to $G$.
Without \eqref{mod:part:followArcsA} and \eqref{mod:part:followArcsB}, it could be possible to find a feasible solution, even when no partition of $G$ into pruned binomial trees exists.

Constraints \eqref{mod:part:followArcsA} and \eqref{mod:part:followArcsB} can be replaced by stronger
\begin{subequations}
\begin{align}
\label{mod:part:followArcsAStrongerA}
y^{v_0}_{is}+y^u_{i s} + \sum\limits_{v\in V\setminus N(u)}y^v_{\ell s}&\leq 1 & u\in V,i\in I,\ell\in C(i), s\in S,  \\
\label{mod:part:followArcsAStrongerB}
y^{v_0}_{is}+y^u_{\ell s} + \sum\limits_{v\in V\setminus N(u)}y^v_{i s}&\leq 1 & u\in V,i\in I,\ell\in C(i), s\in S. 
\end{align}
\end{subequations}

We now use the notion of graph power $G^k=(V,E^k)$ commonly defined as a graph with the same set of nodes as $G$,
and an edge between two nodes in $G^k$ is present iff there is a path of length at most $k$ between them in $G$.
For our purposes, we use a slightly modified definition of the edge set
$$E^k=\{\{u,v\}:\text{there exists a path between $u$ and $v$ in $G$ of length $k$}\}.$$
For a given $k$, let $W$ be a maximum independent set in $G^k$.
Further strengthening can be achieved by introducing valid inequalities
\begin{subequations}
\begin{align}
\label{mod:part:vi}
y^{v_0}_{is}+ \sum\limits_{v\in W}(y^v_{is}+y^v_{\ell s})&\leq 1 & i\in I,\ell\in C^k(i), s\in S,k\in\{1,\dots,\Delta-1\}, 
\end{align}
\end{subequations}
where $\Delta$ denotes the graph diameter of $G$.
Consider indices $i$ and $\ell$ in a binomial tree.
By expanding the summation on the left-hand side of \eqref{mod:part:vi}
we obtain variables such that no two nodes in their superscript can be $i$-th and $\ell$-th node of a binomial tree at the same time.
It is obvious for variables with the same superscript - one node cannot act as both $i$-th and $\ell$-th node of a binomial tree.
That is already guaranteed by \eqref{mod:part:nodeBelongs}.
The indices assume a certain distance $k$ between the nodes with the corresponding labels.
The fact that $W$ is an independent set in the modified graph power ensures that no two nodes between which there is no path of the desired length have these indices.

Another improvement of this model is achieved by a symmetry removal.
If a broadcast tree is identical to a binomial tree, we notice that nodes with labels from $C(i)$, i.e., children of some node $v$ with $\beta(v)=i$ are informed in increasing time steps.
For example in $B^3$, $C(2)=\{4,6\}$ and the corresponding nodes are informed in time step 2 and 3, respectively.
If a label $\ell\in C(i)$ corresponds to a node of a binomial tree that is pruned (if $y^{v_0}_{\ell s}=1$ for some $s\in S$), 
all labels $j\in C(i)$ such that $j>\ell$ can also be pruned. 
Thus, adding 
\begin{subequations}
\begin{align}
\label{mod:part:sr}
y^{v_0}_{js}&\leq y^{v_0}_{\ell s}&i\in I,j,\ell\in C(i), s\in S
\end{align}
\end{subequations}
to the model reduces the set of feasible solutions.

An objective function is naturally lacking in the formulation of the decision problem.
It is nevertheless straightforward to create an ILP model for the corresponding optimization problem.
Let $z_i=1 \Leftrightarrow t=i$ be a new variable and let $\bar{t}$ be an upper bound on the delay in $(G,S)$.
Problem \ref{prob:min} is formulated as follows:
\begin{subequations}
\begin{align}
\notag &\min\sum\limits_{j=0}^{\bar{t}}z_j,\\
\notag \text{s. t. } \\
\notag \eqref{mod:part:nodeBelongs} - \eqref{mod:part:source1},& \eqref{mod:part:followArcsAStrongerA} - \eqref{mod:part:followArcsAStrongerB}, \eqref{mod:part:vi}, \eqref{mod:part:sr},\\
\notag\sum\limits_{v\in V}y^v_{is} & \leq z_{\lceil\log i\rceil} & i\in I,s\in S,\\
\notag\label{mod:part:optdim}y &\in \{0,1\}^{I\times S\times V'}, z \in \{0,1\}^{\bar{t}}.
\end{align}~
\end{subequations}
