\section{Upper bounds}

The algorithms presented in this section iteratively construct broadcast foerst $T_i, i=1,2,\dots$.
The following inexact method is based on the idea of finding maximum cardinality matching in $G\left[V_{T_i},V\setminus V_{T_i}\right]$.
That means that in each iteration, maximum possible number of nodes are informed.
Maximum cardinality matching can be regarded as finding $|V_{T_i}|$ node-disjoint  binomial trees of order at most one with roots in $V_{T_i}$, maximizing the number of edges.
By generalizing this notion, we iteratively look for $|V_{T_i}|$ node-disjoint (pruned) binomial trees of an arbitrary given order $k$ valued between 1 and $n-|S|$.
Even though this problem is NP-hard for $k\geq 2$ \cite{jansen95}, it is expected that the computational time is sound in most practical instances.
After obtaining the set of binomial trees, first $p$ nodes in each tree is selected and added to the broadcast forest.
The parameter $p\in \{1,\dots,2^k\}$ is thus a part of the input.

Alg. \ref{alg:match} describes the process more formally.
Initially, the broadcast forest consists of isolated sources.

\begin{algorithm}[]
\KwData{$G=(V,E), S\subseteq V, k\in \{1,\dots,n-|S|\}, p\in \{1,\dots,2^k\}$}
%$U\leftarrow S$;\tcp{Set of marked nodes}
$T_s=(V_s,A_s), V_s\leftarrow \{s\}, E_s\leftarrow\emptyset$\;
$t\leftarrow 0$\;
\While{$|V_T|<n$} {
	Find a set of pruned binomial trees $B=\{B_1,\dots,B_{|V_T|}\}$ of order at most $k$ with roots in $V_T$ maximizing $|V_B|$, such that $V_{B_i}\neq V_{B_j}$ whenever $i\neq j$\;
	$V_T\leftarrow V_T\cup \{v:v\in V_B:\beta(v)\leq p\}$\;
	$E_T\leftarrow E_T\cup \{\{u,v\}\in E_B: \beta(u)\leq p,\beta(v)\leq p\}$\;
	$t\leftarrow t+1$\;
}
\Return t\;
%\Return $\lceil k/s \rceil$\;
 \caption{A method for determining an upper bound}
\label{alg:genub}
\end{algorithm}



%
%We present a simple general method for determining an upper bound on the optimal solution.
%The idea is to construct broadcast trees in parallel from their roots in sources.
%A new node is connected to a tree is it was not connected to any other tree in any of the previous iterations.
%The steps are summarized in Alg. \ref{alg:genub}.
%
%\begin{algorithm}[]
%\KwData{$G=(V,E), S\subseteq V$}
%$U\leftarrow S$;\tcp{Set of marked nodes}
%$T_s=(V_s,A_s), V_s\leftarrow \{s\}, E_s\leftarrow\emptyset$\;
%\While{$|U|<n$} {
%	\For{$s\in S$} {
%		Select $v\in V\setminus U:\exists u\in V_s \text{ s. t.} \{u,v\}\in E$ according to some strategy\; 	
%		$U\leftarrow U\cup\{v\}$\;
%		$V_s\leftarrow V_s\cup\{v\}$\;
%		$A_s\leftarrow A_s\cup\{u,v\}$\;
%	}
%}
%\For{$s\in S$} {
%$	
%\sigma(v)=\begin{cases}
%		0, \text{ if } v\in L(T_s),\\
%		\max\limits_{k\in 1,\dots,|N^+(v)|}\{k+\sigma(j_k):j_k\in N^+(v)\wedge \forall \ell<k:\sigma(j_\ell)>\sigma(j_k)\},\\
%		\hfill \text{ if }v\in V_s\setminus L(T_s).
%
%\end{cases}\;
%$
%}
%\Return $\max_{s\in S}\sigma(s)$\;
%%\Return $\lceil k/s \rceil$\;
% \caption{A method for determining an upper bound}
%\label{alg:genub}
%\end{algorithm}
%The algorithm consists of two phases. 
%In the first phase (lines 3 -- 10), $m$ broadcast trees are constructed.
%The second phase (lines 11 -- 13) uses a recursive formula that determines in linear time an optimal broadcast time for a subtree of $T_s$ rooted at $v\in V_s$.
%The value  $\sigma(s)$ than gives the optimal broadcast time for $T_s$.
%The complexity of Alg. \ref{alg:genub} depends on the strategy according to which the trees are constructed.
%For common strategies such as BFS and DFS it is $\mathcal{O}(n+|E|)$, for the shortest paths trees $\mathcal{O}(|E| + n\log n)$.
%
%
