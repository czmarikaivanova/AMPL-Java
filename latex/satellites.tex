% This is a general template file for the LaTeX package SVJour3mm
% for Springer journals.          Springer Heidelberg 2010/09/16
%
% Copy it to a new file with a new name and use it as the basis
% for your article. Delete % signs as needed.
%
% This template includes a few options for different layouts and
% content for various journals. Please consult a previous issue of
% your journal as needed.
%
%%%%%%%%%%%%%%%%%%%%%%%%%%%%%%%%%%%%%%%%%%%%%%%%%%%%%%%%%%%%%%%%%%%
%
% First comes an example EPS file -- just ignore it and
% proceed on the \documentclass line
% your LaTeX will extract the file if required
\begin{filecontents*}{example.eps}
%!PS-Adobe-3.0 EPSF-3.0
%%BoundingBox: 19 19 221 221
%%CreationDate: Mon Sep 29 1997
%%Creator: programmed by hand (JK)
%%EndComments
gsave
newpath
  20 20 moveto
  20 220 lineto
  220 220 lineto
  220 20 lineto
closepath
2 setlinewidth
gsave
  .4 setgray fill
grestore
stroke
grestore
\end{filecontents*}
%
\RequirePackage{fix-cm}
%
%\documentclass{svjour3}                     % onecolumn (standard format)
%\documentclass[smallcondensed]{svjour3}     % onecolumn (ditto)
\documentclass[smallextended]{svjour3}       % onecolumn (second format)
%\documentclass[twocolumn]{svjour3}          % twocolumn
%
\smartqed  % flush right qed marks, e.g. at end of proof
%
\usepackage{graphicx}
\usepackage{tkz-graph}
\usepackage{amsmath, amssymb}
\usepackage{graphicx}
\usepackage{tikz}
\usepackage{amsfonts}
\usepackage{subcaption}
\usepackage{hyperref}
%\usepackage{pdflscape} % Landscape pages
\usepackage{color}	% Different font colors
\usepackage{enumerate}
\usepackage{enumitem}
\usepackage{mathtools}
\usepackage{color}
\usepackage[linesnumbered]{algorithm2e}
\SetKwRepeat{Do}{do}{while}

\captionsetup{compatibility=false}

\newtheorem{observation}[theorem]{\textbf{Observation}}

\tikzset{
every edge/.style={fill=none} 
every node/.style={shape=circle,fill=gray!40,draw }}



%\newtheorem{proposition}{Proposition}
%\newtheorem{obsservation}{Observation}
%\newtheorem{corollary}{Corollary}
%newtheorem{lemma}{Lemma}
%\newtheorem{problem}{Problem}
%
% \usepackage{mathptmx}      % use Times fonts if available on your TeX system
%
% insert here the call for the packages your document requires
%\usepackage{latexsym}
% etc.
%
% please place your own definitions here and don't use \def but
% \newcommand{}{}
%
% Insert the name of "your journal" with
% \journalname{myjournal}
%
\begin{document}

\title{Insert your title here%\thanks{Grants or other notes
%about the article that should go on the front page should be
%placed here. General acknowledgments should be placed at the end of the article.}
}
\subtitle{Do you have a subtitle?\\ If so, write it here}

%\titlerunning{Short form of title}        % if too long for running head

\author{First Author         \and
        Second Author %etc.
}

%\authorrunning{Short form of author list} % if too long for running head

\institute{F. Author \at
              first address \\
              Tel.: +123-45-678910\\
              Fax: +123-45-678910\\
              \email{fauthor@example.com}           %  \\
%             \emph{Present address:} of F. Author  %  if needed
           \and
           S. Author \at
              second address
}

\date{Received: date / Accepted: date}
% The correct dates will be entered by the editor


\maketitle

\begin{abstract}
\keywords{}
% \PACS{PACS code1 \and PACS code2 \and more}
% \subclass{MSC code1 \and MSC code2 \and more}
\end{abstract}

\section{Introduction}
\label{intro}
The minimum broadcast time (MBT) problem consists of a set of communication nodes with a subset of source nodes. 
The task is to disseminate a signal to every node in a shortest possible time $t^*$ (delay), while abiding by communication rules.
An \emph{informed} node is a node that has received the signal.
Otherwise, a node is \emph{uninformed}.
At the beginning, the set of informed nodes is exactly the set of sources.
An informed node $u$ can send the signal to an uninformed node $v$ if $u$ and $v$ are located within a communication vicinity of each other.

The time is divided into discrete time steps.
At each time step, every informed node can forward the signal to at most one uninformed neighbor.
The number of informed nodes at some time step can therefore be no more than the double of the number of informed nodes at the previous time step.
%\subsection{Motivation and Related Work}
%This communication protocol differs from various wireless communication models where a signal can be relayed to all nodes within a visibility range of a sender.
This communication protocol appears in various practical application such as communication among computer processors or telephone networks.
%The applications are however not confined to wired networks.
Situations where the signals have to cover large distances typically assume sending the signal to one neighbor at a time.
This is common in satellite communication.

\subsection{Literature overview}

MBT has been shown to be NP-complete on arbitrary graphs in \cite{slater81}. 
The problem remains NP-complete even for several restricted communication networks \cite{jansen95} such as
split graphs with delay $t^*=2$, chordal graphs,
bipartite and planar graphs with maximum degree 3 and a single source,
several variants of grid graphs,
bipartite planar graphs with delay $t^*=2$ and maximum degree 3. 
Yet another NP-completeness result for 3-regular planar graphs with delay 2 is shown in \cite{middendorf93}.
The problem is known to be polynomial in trees \cite{slater81}.
Whether the problem is polynomial or NP-complete for split graphs with a single source was stated as an open questions in \cite{jansen95}, and hasn't been answered yet to our best knowledge.

A number of inexact methods for both general and special graph classes has been proposed in the literature during the last three decades.
One of the first works of this category is \cite{scheuermann84}, 
where the authors introduce an exact dynamic programming algorithm based on generating all maximum matchings in an induced bipartite graph.
Additional contribution of \cite{scheuermann84} are heuristic approaches for near optimal broadcasting.
From more recent works we mention \cite{hasson04}, which describes a meta heuristic algorithm for MBT, and provides a comparison with other existing methods.
The communication model is considered in an existing satellite navigation system in \cite{chu17}, where a greedy inexact methods is proposed together with a non-linear mathematical model.
Examples of additional efficient heuristics can be found e.g. in \cite{harutyunyan06,harutyunyan14,wang10,jimborean13}.

Approximation algorithms for MBT are studied in \cite{kortsarz95}. 
The authors argue that methods presented in \cite{scheuermann84} provide no guarantee on the performance and state an $n$-node wheel as an example of an unfavourable instance.
They introduce an $\mathcal{O}(\sqrt{n})$-additive approximation algorithm for broadcasting in general graphs.
They further provide approximation algorithms for several graph classes with small separators with approximation ratio proportional to the separator size times $\log n$.
Another algorithm with $\mathcal{O}\left(\frac{\log n}{\log \log n}\right)$-approximation\footnote{All logarithms in this paper are of base 2} ratio is given in~\cite{elkin03}.
Most of the works above consider a single source.

A related problem extensively studied in the literature is the minimum broadcast graph \cite{grigni91,mcgarvey16}. 
A broadcast graph supports a broadcast from any node to all other nodes in optimal time $\lceil\log n\rceil$.
For a given $n$, the effort is to determine properties such as the minimum number of edges and the minimum degree in broadcast graph on $n$ nodes.
The authors of \cite{mcgarvey16} study ILP models for $c$-broadcast graphs, a generalization of the problem that allows transmission of the signal to at most $c$ neighbours in a single time step.

\section{Network Model and Notation}

The communication network is represented by a connected graph $G=(V,E)$ and a subset $S\subseteq V$ referred to as the set of sources, where $|V|=n$. 
Let $V\left[F\right]=\{v\in V:\text{ some edge in } F \text{ is incident to } v\}, F\subseteq E$.
For disjoint node sets $V_1,V_S\subseteq V$, we denote by $G\left[V_1,V_2\right]$ the bipartite graph $(V_1\cup V_2,E')$, where  $E'=\{\{u,v\}\in E: u\in V_1,v\in V_2\}$. 

\begin{definition}
The \emph{broadcast time} $\tau(G,S)$ of a node set $S\subseteq V$ in $G$ is defined as: 
\begin{equation*}
\tau(G,S)=
\begin{cases}
	0, S=V,\\
	1+\min\{\tau(G,S\cup V\left[M\right]):M \text{ is a matching in } G\left[S, V\setminus S\right]\}, S\neq V.
\end{cases}
\label{eq:btime}
\end{equation*}
\end{definition}
%Broadcasting is defined as a sequence of sets $(S=V_0\subseteq\dots\subseteq V_k = V)$ where each $V_i$ represents the nodes informed after time step $i$, $0\leq i\leq t$.
%For each node $v\in V_i\setminus V_{i-1}$, there exists a single node $\pi(v)\in V_{i-1}$ adjacent to $v$, which forwarded the signal to $v$.
%Also, for every two nodes $u,v\in V_i\setminus V_{i-1}$ we have $\pi(u)=\pi(v)$.
%The value $k$ is referred to as \emph{delay}.
The optimization problem in question is formulated as follows \cite{jansen95,middendorf93}:
\begin{problem}\label{prob:min}
Given $G=(V,E)$ and $S\subseteq V$, find a sequence $(S=V_0\subseteq\dots\subseteq V_{\tau(G,S)}=V)$ and a function $\pi:V\setminus S\to V$, such that for each $v\in V\setminus S:\{v,\pi(v)\}\in E$, and for each $u,v\in V_i\setminus V_{i-1}: \pi(u)=\pi(v)\Leftrightarrow u=v$.
\end{problem}

Given a feasible sequence of node sets and $\pi$, we derive the following broadcast trees \cite{grigni91}:
For $s\in S$, a broadcast tree $T_s=(V_{T_s},E_{T_s})$ is a time-labeled directed subgraph of $G$ describing a broadcast originated in $s$ by the rules:
\begin{enumerate}
\item $T_s$ is rooted at $s$ with arcs directed towards the leaves.
\item each node $v\in V_i$ is labeled with an integer $\tau(v)=i$,
\item for $v\in V\setminus S, v\in V_{T_s}\Leftrightarrow \pi(v)\in V_{T_s}$, 
\item $E_{T_s}=\{\{u,v\}: u,v\in V_{T_s}, \pi(v)=u\}$, and
\item for two sources $s,s'\in S, s\neq s'$ we have that $V_{T_s}\cap V_{T_{s'}}=\emptyset$. 
\end{enumerate}

Any feasible sequence of node sets and $\pi$ yield a set of trees $T=\{T_s:s\in S\}$ that forms a partition of $G$ into trees referred to as \emph{broadcast forest}.
The number of leaves leaves in $T_s$ is denoted by $L(T_s)$.
The function $\alpha:V\to S$ associates a node $v$ with the source in which the tree containing $v$ is rooted.

Given a broadcast forest $T$, the delay is determined as $\max_{v\in V}\{\tau(v)\}$.
%A set of trees $T=\{T_s:\cup_{s\in S}V_{T_s}=V\}$ forms a partition of $G$ into trees referred to as \emph{broadcast forest}.
%$T$ can be derived from a given sequence of sets of nodes defining broadcasting and the mapping $\pi$.
For any integer $i$, let $T^i_s$ be the subtree of $T_s$ obtained by pruning all nodes $v\in V_{T_s}$ with $\tau(v)>i$.
Analogously, we define $T^i=\{T^i_s:s\in S\}$. 


%We also define the set $A=\{(i,j),(j,i):\{i,j\}\in E\}$ that consists of all arcs that can be derived by directing edges in $E$.
The degree of node $v$ in $G$ is denoted by $\text{deg}_G(v)$.
If $G$ is directed, $\text{deg}^-_G(v)$ and $\text{deg}^+_G(v)$ denote in-degree and out-degree, respectively.
For the diameter of $G$ we use the symbol $\Delta_G$.
Whenever there is no danger of confusion, the subscript $G$ is omitted.
The set of neighbors of $v\in V$ in $G$ is denoted by $N(v)$ and $N^+(v)$ in directed graphs.

%For convenience, we consider the following definition of broadcast trees \cite{grigni91}:
%For $s\in S$, a broadcast tree $T_s$ with node set $V_{T_s}$ is a time-labeled directed subgraph of $G$ describing a broadcast originated by $s$ by the following rules:
%\begin{enumerate}
%\item $T_s$ is rooted at $s$ with arcs directed towards the leaves.
%\item Each node $v$ is labeled with an integer $\tau(v)$, where $\tau(s)=0$.
%\item Whenever $v$ is a parent of $u$ in $T_s$ $\tau(v)<\tau(u)$.
%\item Whenever $v$ and $u$ are siblings in $T_s$, $\tau(v)\neq \tau(u)$. 
%\end{enumerate}
%The number of leaves in $T_s$ is denoted by $L(T_s)$.
\section{Exact methods}

Problem \ref{prob:min} can be formulated as an integer linear program. % and solved to optimality. 
In this section, we present two different modeling approaches. 

\subsection{Broadcast time model}
The first studied model is a straightforward formulation of the problem.
Consider variables 
$$ x_{uv}^k=
\begin{cases} 
1, \text{ if } v\in V_k \text{ and } \pi(v)=u,\\ 
0, \text{ otherwise},
\end{cases}
z_{k}=\begin{cases}
1, \text{ if } k\leq\tau(G,S),\\
0, \text{ otherwise},
\end{cases}
$$
and a variable $t^*$ representing the number of necessary time steps.
The worst case scenario is when $G$ is a path $v_1,\dots,v_n$ with $S=\{v_1\}$. 
In such an instance, the necessary number of time steps is $n-1$, which gives a trivial upper bound $\bar{t}=n-s$ on the value of $t^*$.
Problem \ref{prob:min} is then formulated as follows: 
\begin{subequations}\label{mod:basic}
\begin{align}
\label{mod:basic:obj} \min \sum\limits_{k=1}^{\bar{t}}z_k \\ 
%\label{mod:basic:onefromroot} \sum_{u \in N(s)}x^1_{su} & \leq 1 & s\in S,\\
\label{mod:basic:singlein} \sum\limits_{k=1}^{\bar{t}}\sum\limits_{v\in N(u)}x_{vu}^k & = 1 & u\in V \setminus S,\\
\label{mod:basic:uniqueTout} \sum\limits_{v\in N(u)}x_{uv}^k & \leq 1  & k=1,\dots,\bar{t},u\in V,\\
%\label{mod:basic:tIncreases} \sum\limits_{v\in N(u)}x_{uv}^k &\leq\sum\limits_{\ell=1}^{k-1}\sum\limits_{w\in N(u)\setminus\{v\}} x_{wu}^{\ell}  & u\in V\setminus S, k=2,\dots,\bar{t},\\
\label{mod:basic:tIncreases} x_{uv}^k &\leq\sum\limits_{\ell=1}^{k-1}\sum\limits_{w\in N(u)\setminus\{v\}} x_{wu}^{\ell}  & \{u,v\}\in V\setminus S, k=2,\dots,\bar{t},\\
\label{mod:basic:tcrel} \sum\limits_{k=1}^{\bar{t}}k\cdot x_{uv}^k & \leq t^* &  (u,v)\in A,\\
%\label{mod:basic:tcrel} \sum\limits_{t=1}^{n-1}t\sum\limits_{j\in N(i)}x_{ij}^k & \leq c &  i\in V,\\
\label{mod:basic:positiveCost}x_{uv}^1 & = 0 & (u,v)\in A, u \in V\setminus S,\\
\label{mod:basic:dim}&&x \in \{0,1\}^{A\times V},t^*\in\{1,\dots,\bar{t}\}.
\end{align}~
\end{subequations}
%Constraints \eqref{mod:basic:onefromroot} indicate that for each source node $s$, there is at most one adjacent node $u\in V_1$ such that $\pi(u)=s$.
By \eqref{mod:basic:singlein}, for every non-source node $u$, there is exactly one node $v$ such that $\pi(u)=v$.
Constraints \eqref{mod:basic:uniqueTout} enforce that for each node $u\in V$ and each subset $V_k$, there is at most one adjacent node $v\in V_k$ with $\pi(v)=u$.
The requirement that a non-source node has a neighbor $v\in V_k$ such that $\pi(v)=u$ only if there exists a node $w\in V_{k-1}$ such that $\pi(u)=w$ is modeled by \eqref{mod:basic:tIncreases}. 
%The requirement that only informed nodes can relay a signal is modeled by \eqref{mod:basic:tIncreases}. 
%The maximum time step at which any transmission takes place is captured by \eqref{mod:basic:tcrel}, and finally, \eqref{mod:basic:positiveCost} states that a node that is not a source never transmits in the first time step.
The length of the sequence of subsets is captured by \eqref{mod:basic:tcrel}, and finally, \eqref{mod:basic:positiveCost} state that if $\pi(v)\not\in S$ for some $j\in V$, then $v\not\in V_1$.
\subsection{Binomial tree model}

The binomial tree $B^k$ of order $k$ is an ordered tree defined recursively as follow \cite{cormen90}:
\begin{itemize}
\item The binomial tree $B^0$ consists of a single node.
\item The binomial tree $B^k$ has a root with $k$ children where the $i$-th child is the root of a binomial tree of order $k-i$, $i=1,\dots,k$.
\end{itemize}
An example of $B^3$ is depicted in Fig.\ref{fig:beta}.
%If a solution to MBT consists of broadcast trees that are binomial, the number of informed nodes doubles in each time step.
For a given time step $k$, the maximum number of informed nodes within $k$ steps is $|S|2^k$.
This occurs when the solution of MBT consists of broadcast trees that are binomial.
%Any broadcast tree can be regarded as pruned binomial tree.  
%Problem \ref{prob:min} can therefore be restated as finding a partition of $G$ into $m$ pruned binomial trees 
\begin{observation}
\label{obs:btspread}
if $r\in S$ is the root of $B^k$, then $\tau(B^k,\{r\})=k$.
\end{observation}
\begin{figure}
\centering
\begin{tikzpicture}[->,scale=.7,every node/.style = {scale=.6,draw,shape=circle, align=center, fill=gray!30}, level/.style={sibling distance=2.5cm/#1,level distance=1.0cm}]]
   \node[] {1}
   	   child[] { node {2} 
	   	   child {node {4}
		  	child {node {8}} 
		   }
		   child {node {6} }
	   }
   child[] { node {3}
   	   child { node {7} }
	   }
   child[] { node {5}
	}
 ;
\end{tikzpicture}
\caption{A binomial tree with nodes labeled by their $\beta$-positions}
\label{fig:beta}
\end{figure}

\subsubsection{Binomial trees over positive integers}

Let $k\in \mathbb{N}$ and $I=\{1,\dots,2^k\}$. 
A directed binomial tree $B^k=(V_{B^k},A_{B^k})$ with arcs oriented towards the leaves has a regular structure that allows to define a systematic numbering of nodes so that a node number determines unambiguously a position in $B^k$.
That is, we need an applicable bijective function $\beta:V_{B^k}\to I$.
A suitable bijection $\beta$ assigns values from $I$ to nodes increasingly with decreasing outgoing degree.
If there is an ambiguity, a node whose parent has a lower number is assigned a lower number.
This function is defined recursively as
\begin{equation*}
\label{eq:beta}
\beta(v)=\begin{cases}
1,\text{ if } v \text{ is the root of } B^k,\\
\beta(u) + 2^{k-deg^+(v)-1}, (u,v)\in A_{B^k},\text{ otherwise}.
\end{cases}
\end{equation*}
Nodes in Fig. \ref{fig:beta} are labeled with their $\beta$-values.

Pairs of integers that are assigned to adjacent nodes in a binomial tree are defined by relation
\begin{equation*}
\label{eq:betarel}
R=\{(i,j)\in\mathbb{N}\times\mathbb{N}:j=i+2^k,k\geq\log i,k\in \mathbb{Z}\}.
\end{equation*}

We further define the function $\pi':\mathbb{N}\to\mathbb{Z}_+$ that for a given integer $i$ associated with node $v$
determines $\beta$-value of parent of $v$: 
\begin{align*}
\label{eq:piprime}
\pi'(1)&=0,& \\
\pi'(j)&=j-2^{\lceil\log j\rceil -1}, &j > 1.
\end{align*}
Using functions $\alpha$, $\beta$, $\pi'$ and the relation $R$, we introduce the notion of binomial trees over positive integers.
\begin{definition}
A pair $(I,X)$ where $I=\{1,\dots,2^k\}$, $k\in\mathbb{Z}_+$, $(\alpha(u),\beta(u))=(\alpha(v),\beta(v))\Leftrightarrow u=v$ and
$X=\{(i,j)\in R, i,j\in I\}$ is an \emph{integer binomial tree of order $k$}. 

$(I,X)$, where $I\subseteq\{1,\dots,2^k\}, k\in \mathbb{Z}_+,\pi(j)\in I$ for all $j\in I\setminus\{1\}$ and
$X=\{(i,j)\in R, i,j\in I\}$ is an \emph{integer binomial tree of order $k$ pruned at $\{1,\dots,2^k\}\setminus I$}.
\end{definition}
We now develop a method for modelling Problem \ref{prob:min} whose main principle is finding a partition of $G$ into pruned binomial trees.
For any integer $i$, let $I_i=\{2^{i-1}+1,\dots,2^i\}$.
%\begin{problem}
%\label{prob:dec}
%Given $G=(V,E)$, $S\subseteq V$ and $t\in \mathbb{N}$, is there a sequence $S=V_0\subseteq\dots\subseteq V_t=V$ 
%and a mapping $\pi:V\setminus S\to V$, such that for each $v\in V\setminus S:\{v,\pi(v)\}\in E$ and for each  $u,v\in V_i\setminus V_{i-1}: \pi(u)\neq \pi(v)\Leftrightarrow u\neq v$?
%\end{problem}
%For a delay $t$, at most $s\cdot 2^t$ nodes can be informed within $t$ steps. 
%Therefore, we assume $n\leq 2^ks.
%This can be achieved when the broadcast forest $T$ consists of binomial trees $B^t$ of order $t$ rooted at sources $s\in S$.
%Hence, if there is a partition of $G$ into $s$ pruned binomial trees of order at most $t$ rooted at sources, then $(G,S,t)$ is a YES instance of Problem \ref{prob:dec}.
%Finding a partition of $G$ into $s$ pruned binomial trees can be equivalently formulated as finding a partition of $G''$ into $s$ (complete) binomial trees, 
%where $G''$ is constructed from $G$ as follows:
%Let $\alpha\coloneqq s\cdot 2^k-|V|$, and let $K_\alpha=(V_\alpha,E_\alpha)$ be a complete graph on $\alpha$ nodes.
%Each node in $K_\alpha$ is connected to every node $v\in V$ in the original graph $G$.
%Thus, $G''=(V'',E'')$ with $V''=V\cup V_\alpha$ and $E''=E\cup E_\alpha\cup \{\{u,v\}: u\in V \wedge v\in V_\alpha\}$.
%The set of arcs $A''$ is constructed by creating two arcs of opposite orientation for each edge, but arcs with orientation from $V_\alpha$ to $V$ are excluded. 
%Formally, $A''=A\cup\{(u,v),(v,u): \{u,v\}\in E_\alpha\}\cup\{(u,v):u\in V \wedge v\in V_\alpha\}$.

%\begin{observation}\label{obs:deg}
%For each $i\in\{1,\dots,t\}$, the set $\{v\in V^t: 1\leq\beta(v)\leq2^i\}\setminus\{v\in V^t:1\leq\beta(v)\leq2^{i-1}\}$ contains nodes with out-degree $t-i$.
%\end{observation}
%\begin{observation}\label{obs:childdeg}
%Children of $v\in V^t$ with $\text{deg}^+(v)=\ell$ have out-degree $0,\dots,\ell-1$.
%\end{observation}
\begin{observation}\label{obs:eqbetalevel}
$\pi'$ is injective.
%For $i_1,i_2\in I_i$, $\pi'(i_1)=\pi'(i_2)\Leftrightarrow i_1=i_2$.
\end{observation}
\begin{proof}
We notice that $\lceil\log (2^{i-1}+1)\rceil=\dots =\lceil\log 2^i\rceil$, and thus $\pi'(2^{i-1}+1),\dots,\pi'(2^i)$ are pairwise different.\qed 
\end{proof}

\begin{proposition}\label{lem:probeq}
A graph $G=(V,E)$ and $S\subseteq V$ has $\tau(G,S)\leq k$ iff it is possible to assign integers to nodes
such that they form $m$ (pruned) integer binomial trees of order $k$ or smaller rooted at sources.
\end{proposition}
\begin{proof}
Assume nodes in $V$ can be labeled by integers from $I=\{1,\dots,2^k\}$ so that they form (pruned) integer integer binomial trees $B_\ell$ of order at most $k$, $1\leq\ell\leq m$.
For $r\in V_{B_\ell}$ with $\alpha(r)=r$ and $\beta(r)=1$, by Obs. \ref{obs:btspread}, $\tau(B_\ell,\{r\})\leq k$.
The sequence of node sets $V_0,\dots,V_k$ is constructed by setting $V_0=\{v\in V:\beta(v)=1\}$, and $V_{i}=V_{i-1}\cup\{v\in V: \beta(v)\in I_i\}$.
Moreover, $\pi(v)=u\Leftrightarrow \alpha(v)=\alpha(u)~\&~\pi'(\beta(v))=\beta(u)$.
Also, $\pi(u)=\pi(v)\Leftrightarrow u=v$ follows from Obs. \ref{obs:eqbetalevel}, because nodes in $V_i$ have $\beta$ values in $I_i$.

Conversely, suppose there is a sequence of subsets and a function $\pi$ in $G$ with the desired properties.
Nodes are associated with integers from $I$ assigned according to the following steps:
\begin{enumerate}
\item $\beta(s)=1$ for all $s\in V_0$,
\item For $v\in V_i$, set $\beta(v)=j$ such that $j\in I_i$ and $(\pi'(\beta(v)),j)\in R, 1\leq i\leq k$.  
\end{enumerate}
\qed
\end{proof}
\subsubsection{The formulation}
Consider a graph $G'=(V',E')$ constructed  by adding a universal node $v_0$ to $G$. 
The set of nodes and edges is then $V'=V\cup \{v_0\}$ and $E'=E\cup\{\{v_0,v\}:v\in V\}$.
The ILP model based on partition into binomial trees uses variables
$$
y_{is}^v=\begin{cases}
1, \text{ if } \beta(v)=i \text{ and } \alpha(v)=s,\\ 
0, \text{ otherwise},\\
\end{cases}
z_{j}=\begin{cases}
1, \text{ if } j\leq\tau(G,S),\\
0, \text{ otherwise},
\end{cases}
$$
where $v\in V'$, $i\in I$, $s\in S$ and $0\leq j\leq \bar{t}$. 
With the definition of $G'$ above, it is straightforward to specify constraints that enforce desired values for $y$-variables.
Whenever $y_{is}^{v_0}=1$, it indicates that the binomial tree $B^t_s$ is pruned at node with position $i$.
%An obvious weakness of this approach is  that the number of nodes increases to $|V''|=\mathcal{O}(ns)$, and the dimension of variables is thus $\mathcal{O}(n^2s^2)$.
%However, once a suitable partition is found, the arcs of binomial trees contained in $K_\alpha$ can be diversely shuffled while preserving the layour of binomial trees in $G$.
%Instead of adding the entire complete graph $K_\alpha$, a single node $v_0$ with a loop $(v_0,v_0)$ is connected as an apex to the original $G$.
%Let us denote this multigraph as $G'=(V',E')$, where $V'=V\cup\{v_0\}$, $E'=E\cup\{\{u,v_0\}:u\in V\}\cup\{\{v_0\}\}$. 
%The arc set is then analogously defined as $A'=A\cup\{(u,v_0): u\in V\}\cup\{(v_0,v_0)\}$.
%The requirement for partition into binomial trees has to be adjusted accordingly.
%The subtrees contained in $G$ remain unchanged, every arc $(u,v)\in A^k_i, i=1,\dots,s$ in $G''$ with $u\in V$ and $v\in V_\alpha$ becomes $(u,v_0)$ in $G'$,
%and every $(u,v)\in A_\alpha$ becomes $(v_0,v_0)$.
%So, $v_0$ acts as a universal node that can substitute several nodes in each binomial tree.
Let us define the set $C(i)$ of $\beta$-positions of children (direct descendants) of node $v$ with $\beta(v)=i$ in a binomial tree of order $k$:
\begin{equation}
\label{eq:c1}
C(i)=\{2^j+i:j=\lceil\log_2 i\rceil,\dots,k-1\}.
\end{equation}
The formulation based on binomial trees is the following:
\begin{subequations}\label{mod:partition}
\begin{align}
\notag &\min\sum\limits_{j=0}^{\bar{t}}z_j,\\
\notag \text{s. t. } \\
\label{mod:part:nodeBelongs} \sum\limits_{i\in I}\sum\limits_{s\in S}y^v_{is} & = 1 & v\in V,\\
\label{mod:part:treeHasIJ} \sum\limits_{v\in V'}y^v_{is} & = 1 & i\in I,s\in S,\\
\label{mod:part:source1} y_{1s}^s & = 1  & s\in S,\\
%\label{mod:part:noReturn} y^u_{ij}+y^v_{lj} &\leq 1 & i\in I,l\in C(i), j\in J, u\in V_\alpha,v\in V,\\
%\label{mod:part:followArcs} y^u_{is}+y^v_{\ell s} &\leq 1 & i\in I,\ell\in C(i), s\in S, u,v\in V',(u,v)\not\in A',\\
%\label{mod:part:followArcsA} y^u_{is}+y^v_{\ell s} &\leq 1 & i\in I,\ell\in C(i), s\in S, u,v\in V,\{u,v\}\not\in E,\\
%\label{mod:part:followArcsB} y^{v_0}_{is}+y^v_{\ell s} &\leq 1 & i\in I,\ell\in C(i), s\in S, v\in V,\\
\label{mod:part:followArcsA}y^{v_0}_{is}+y^u_{i s} + \sum\limits_{v\in V\setminus N(u)}y^v_{\ell s}&\leq 1 & u\in V,i\in I,\ell\in C(i), s\in S,  \\
\label{mod:part:followArcsB}y^{v_0}_{is}+y^u_{\ell s} + \sum\limits_{v\in V\setminus N(u)}y^v_{i s}&\leq 1 & u\in V,i\in I,\ell\in C(i), s\in S,\\ 
\label{mod:part:yzrel}\sum\limits_{v\in V}y^v_{is} & \leq z_{\lceil\log i\rceil} & i\in I,s\in S,\\
\label{mod:part:dim}&&y \in \{0,1\}^{I\times S\times V'}, z\in \{0,1\}^{\bar{t}}.
\end{align}~
\end{subequations}

%As the model represents the decision problem, it suffices to find any feasible solution, and no objective function is needed.
The interpretation of constraints \eqref{mod:part:nodeBelongs} is that every node in the original graph $G$ belongs to exactly one binomial tree.
Note that these constraints are quantified only over $V$ and not over $V'$.
In this way it is achieved that $v_0$ can be regarded as a part of several binomial trees.
By \eqref{mod:part:treeHasIJ} is ensured that exactly one node, possibly $v_0$, is allocated to position $i$ of each binomial tree.
By the summation over $V'$ is ensured, that pruned nodes are collectively represented by $v_0$.
Next, \eqref{mod:part:source1} enforce that source nodes are always the first nodes in corresponding binomial trees, in accordance with definition \eqref{eq:beta} of the function $\beta$.
The remaining two sets of constraints guarantee that the arcs of binomial trees follow edges in $E$.
In particular, it is enforced by \eqref{mod:part:followArcsA} that if $u$ and $v$ are not adjacent in $G$, then $v$ must not act as a child of $u$ in any binomial tree.
Constraints, \eqref{mod:part:followArcsB} forbid any node from $V$ to be a child of $v_0$ in any binomial tree. 
This reflects the obvious fact that if a tree is pruned at some node, all its descendants must also be excluded from the tree.
%The definition of $A'$ also prevents arcs of the binomial trees to be oriented from $V_\alpha$ to $V$.
%In other words, once the signal leaves the original graph $G$ and enters $v_0$, it cannot return back to $G$.
Without \eqref{mod:part:followArcsA} and \eqref{mod:part:followArcsB}, it could be possible to find a feasible solution, even when no partition of $G$ into pruned binomial trees exists.
Finally, the relation \eqref{mod:part:yzrel} between $y$ and $z$ variables follows from Obs. \ref{obs:btspread}. 
It says that whenever there is a node in a position $i$, then the delay is at least $\lceil\log i\rceil$.
%Constraints \eqref{mod:part:followArcsA} and \eqref{mod:part:followArcsB} can be replaced by stronger
%\begin{subequations}
%\begin{align}
%\label{mod:part:followArcsAStrongerA}
%y^{v_0}_{is}+y^u_{i s} + \sum\limits_{v\in V\setminus N(u)}y^v_{\ell s}&\leq 1 & u\in V,i\in I,\ell\in C(i), s\in S,  \\
%\label{mod:part:followArcsAStrongerB}
%y^{v_0}_{is}+y^u_{\ell s} + \sum\limits_{v\in V\setminus N(u)}y^v_{i s}&\leq 1 & u\in V,i\in I,\ell\in C(i), s\in S. 
%\end{align}
%\end{subequations}
%
\subsubsection{Valid inequalities}
Let $W$ be a maximal independent set in $G$.
Model \eqref{mod:partition} is strengthened by 
\begin{align}
\label{mod:part:vibasic}
y^{v_0}_{is}+ \sum\limits_{v\in W}(y^v_{is}+y^v_{\ell s})&\leq 1 & i\in I,\ell\in C(i), s\in S, 
\end{align}
which exploits the fact that no pair of nodes in $W$ is adjacent, and so there must be no two nodes with adjacent $\beta$-positions.

We now generalize this idea by using the notion of graph power $G^m=(V,E^m)$ commonly defined as a graph with the same set of nodes as $G$,
and an edge between two nodes in $G^m$ is present iff there is a path of length at most $m$ between them in $G$.
For our purposes, we use a slightly modified definition of the edge set
$$E^m=\{\{u,v\}:\text{there exists a path between $u$ and $v$ in $G$ of length $m$}\}.$$
Definition \eqref{eq:c1} can be generalized to descendants of an arbitrary distance $m$ from $v$ in $B^k$:
\begin{equation}
C^{m+1}(i)=\bigcup_{j\in C^1(i)}C^m(j).
\end{equation}
For a given $m$, let $W_m$ be a maximal independent set in $G^m$.
Further strengthening of model \eqref{mod:partition} is achieved by introducing valid inequalities
\begin{align}
\label{mod:part:vigeneral}
y^{v_0}_{is}+ \sum\limits_{v\in W_m}(y^v_{is}+y^v_{\ell s})&\leq 1 & i\in I,\ell\in C^m(i), s\in S,1\leq m\leq \Delta_G-1. 
\end{align}
Clearly, inequality \eqref{mod:part:vibasic} is included in \eqref{mod:part:vigeneral} for $m=1$.
The distance between positions $i$ and $\ell=C^m(i)$ in a binomial tree is $m$.
The maximal independent set $W_m$ contains nodes such that length of any path between any two nodes is different from $m$, 
and so there must not be two nodes in $W_m$ with positions $i$ and $\ell$ at the same time.

\subsubsection{Symmetry removal}
Another improvement of this model is achieved by a symmetry removal.
If a broadcast tree is identical to a binomial tree, we notice that nodes with labels from $C(i)$, i.e., children of some node $v$ with $\beta(v)=i$ are informed in increasing time steps.
For example in $B^3$, $C(2)=\{4,6\}$ and the corresponding nodes are informed in time step 2 and 3, respectively.
If a label $\ell\in C(i)$ corresponds to a node of a binomial tree that is pruned (if $y^{v_0}_{\ell s}=1$ for some $s\in S$), 
all labels $j\in C(i)$ such that $j>\ell$ can also be pruned. 
Thus, adding 
\begin{align}
\label{mod:part:sr}
y^{v_0}_{js}&\leq y^{v_0}_{\ell s}&i\in I,j,\ell\in C(i), s\in S
\end{align}
to the model reduces the set of feasible solutions.

%An objective function is naturally lacking in the formulation of the decision problem.
%It is nevertheless straightforward to create an ILP model for the corresponding optimization problem.
%Let $z_i=1 \Leftrightarrow t=i$ be a new variable and let $\bar{t}$ be an upper bound on the delay in $(G,S)$.
%Problem \ref{prob:min} is formulated as follows:
%\begin{subequations}
%\begin{align}
%\notag &\min\sum\limits_{j=0}^{\bar{t}}z_j,\\
%\notag \text{s. t. } \\
%\notag \eqref{mod:part:nodeBelongs} - \eqref{mod:part:source1},& \eqref{mod:part:followArcsAStrongerA} - \eqref{mod:part:followArcsAStrongerB}, \eqref{mod:part:vi}, \eqref{mod:part:sr},\\
%\notag\sum\limits_{v\in V}y^v_{is} & \leq z_{\lceil\log i\rceil} & i\in I,s\in S,\\
%\notag\label{mod:part:optdim}y &\in \{0,1\}^{I\times S\times V'}, z \in \{0,1\}^{\bar{t}}.
%\end{align}~
%\end{subequations}


\section{Lower bounds}
In this section, we study lower bounds on the delay for several restrictions of input graphs.
%An optimal solution is obtained by solving a sequence of decision problems with varying $t$. 
%It is therefore desirable to determine tight lower and upper bounds in order to arrive in the optimum after solving as few decision problems as possible.
Obvious bounds for a general graph instance are given by
\begin{observation}
For an instance $(G,S)$ of Problem \ref{prob:min},
$$\left\lceil\log\frac{n}{|S|}\right\rceil\leq t^* \leq n-|S|.$$
\end{observation}

In the following, we use $m$-step Fibonacci numbers \cite{noe05}, a generalization of well known (2-step) Fibonacci numbers, defined by letting, 
$F^{(m)}_k=0$ for $k\leq 0$, $F^{(m)}_1=1$, and 
other terms according to linear recurrence relation 
\begin{align*}
F^{(m)}_k &=\sum\limits_{i=1}^m F^{(m)}_{k-i}, &\text{ for } k\geq 2.
\end{align*}

Consider a $d$-regular graph with a unique source $s$.
Any broadcast forest consists of a single tree $T_s$.
We investigate the number of leaves in $T_s$, and derive a lower bound on the delay for this graph class.

If the orientation of arcs in $T_s$ is disregarded, $L(T^1_s)=L(T^2_s)=2$.
For $i\geq 3$, $L(T^i_s)$ equals the number of nodes with degree $1,\dots,d-1$ in $T^{i-1}_s$, 
because in a $d$-regular graph, only nodes with degree lower than $d$ can inform new uninformed nodes.
It can also be interpreted as the sum of the number of leaves in $T^{i-d+1}_s,\dots,T^{i-1}_s$, which leads to %the following formula
\begin{equation*}
\label{eq:leafrec}
L(T^i_s)=\sum\limits_{j=i-d+1}^{i-1} L(T^j_s).
\end{equation*}  
This formula equals to the recursive definition of Fibonacci sequence of order $d-1$.
As each of the two base cases, $L(T^1_s)$ and $L(T^2_s)$, equals the double of the base cases of the Fibonacci sequence, the number of leaves in time step $i$ is calculated as
\begin{equation*}
\label{eq:fibleaf}
L(T^i_s)=2 F^{(d-1)}_i.
\end{equation*}  
The number of nodes in $T^i_s$ can be expressed as the sum of nodes newly informed in time steps $1,\dots,i$, that is, the sum of leaves in $T^1_s,\dots,T^i_s$. Thus,
\begin{equation}
\label{eq:fibcnt}
|V_{T^i_s}|=|V_i|=2\sum\limits_{j=1}^i F^{(d-1)}_j.
\end{equation}

\begin{proposition}
For a $d$-regular graph on $n$ nodes and $|S|$ sources, a lower bound on the delay is 
\begin{equation*}
\label{lem:lbreg1}
\underline{t}=\left\lceil\min\{k:2\sum\limits_{j=1}^k F^{(d-1)}_j\geq n\}/|S|\right\rceil.
\end{equation*}
\end{proposition}
\begin{proof}
In order to inform $n$ nodes in the best possible scenario, the signal has to be relayed sufficiently many time steps so that the maximum possible number of informed nodes becomes $n$.
The maximum number of nodes informed within $i$ time steps is given by \eqref{eq:fibcnt}.
We therefore need to set the upper limit of the summation in \eqref{eq:fibcnt} so that the right-hand side exceeds $n$.
The reason why the result is divided by $|S|$ is that the best case scenario with several source nodes assumes that the signals initiated in individual sources are spread evenly.
\qed
\end{proof}

An additional knowledge of a degree sequence of $G$ can be exploited. 
\begin{lemma}
\label{lemma:degorder}
Let $(d_1,\dots,d_n)$ be a degree sequence and $t\leq \bar{t}-|S|$ a given delay.
The maximum number of nodes informed within $t$ time steps is achieved if nodes become informed in the order of their decreasing degree.
\end{lemma}
\begin{proof}
A node with degree $d_i$ informed in $i$-th time step can inform at most $\min\{d_i-1,t-i\}$ nodes.
Consider $l,k$ such that $1\leq k < \ell\leq n$ and nodes $v_k$, $v_\ell$ with $\deg(v_k)=d_k$ and $\deg(v_\ell)=d_\ell$.
Then, $d_k-1>d_\ell-1$ and $t-k > t-\ell$.
If nodes are informed in the order of their decreasing degree, let $K_1$ and $L_1$ be the number of nodes that can be informed by $v_k$ and $v_l$, respectively:
$$
K_1=\min\{d_k-1,t-k\}, ~~~ L_1=\min\{d_\ell-1,t-\ell\}.
$$
If the order in which $v_k$ and $v_l$ are informed is switched, i.e. when $v_k$ and $v_\ell$ is informed in the $\ell$-th and $k$-th time step, respectively, 
we use $K_2$ and $L_2$ to denote the maximum number of nodes informed by $v_k$ and $v_\ell$:
$$
K_2=\min\{d_k-1,t-\ell\} ~~~ L_2=\min\{d_\ell-1,t-k\}.
$$
We now investigate what possible values can the expression 
\begin{equation}
\label{eq:degbounds}
(K_1+L_1)-(K_2+L_2)
\end{equation}
attain.
Observe that $K_1\geq L_2$ and $K_2\geq L_1$. 
If $K_2>L_1$, then $K_2=K_1=d_k-1$.
But in that case, $L_2=L_1=d_\ell-1$. 
We conclude that \eqref{eq:degbounds} always takes a non-negative value.
Therefore, if the nodes are informed in the order of their decreasing degrees,
the number of informed nodes within the give time step is no worse than when the nodes are informed in any other order.
\qed
\end{proof}

Alg.~\ref{alg:dreg} calculates a lower bound on the delay when given number of nodes, sources and a degree sequence in $G$ as an input.
The algorithm iteratively updates possible node degrees in $T^k$ in each time step $k$, and records the maximum possible number of nodes in $V_k$.
For the purpose of finding lower bounds, it is assumed that each node $v\in V_k$ with $\deg_{T^k}(v)<\deg_G(v)$ informs a new uninformed node.
The number of iterations is then the lowest possible delay for given input.
%The main idea is that in the most optimistic case, each node $v\in V_i$ with $\text{deg}_{F_i}(v)<\text{deg}_G(v)$ informs new not yet informed node.
%For the purpose of finding lower bounds, we can assume without loss of generality that $\text{deg}_{F_i}(u)\leq\text{deg}_{F_i}(v) \Leftrightarrow \text{deg}_G(u)\geq \text{deg}_G(v)$.
Once a node $v$ reaches its maximum degree, i.e., when $\text{deg}_{T^k}=\deg_G(v)$ for some $k$, $v$ does not inform any other node in the next time steps.

%For each $i=1,\dots d$, Alg. \ref{alg:dreg} keeps the  number of nodes with degree $i$.
%These values are updated iteratively using dynamic programming until the number of informed nodes reaches $n$. 

\begin{algorithm}
\KwData{$n,m,d_1,\dots,d_n\in \mathbb{N}, m\leq n,\newline 1\leq d_n\leq\dots\leq d_2\leq d_1$}
$a_1\leftarrow\dots\leftarrow a_{m}\leftarrow 1$\;%\tcp{this is a comment}
$a_{m+1}\leftarrow\dots\leftarrow a_{n}\leftarrow 0$\;%\tcp{this is a comment}
$c\leftarrow m$;%\tcp{every source informs a new node}
$~k\leftarrow 0$\;
\While {$c<n$} {
$k\leftarrow k+1$\;
$c_n\leftarrow 0$\;
\For{$i=1,\dots,c$} {
	\If {$a_i<d_i$} {
		$a_i\leftarrow a_i + 1$\;
		$c_n\leftarrow c_n + 1$\;
		\If {$c+c_n\leq n$} {
			$a_{c+c_n}\leftarrow 1$;\tcp{Newly informed node} 
		}
	}
}
$c\leftarrow c + c_n$\;
}
\Return $k$\;
%\Return $\lceil k/s \rceil$\;
 \caption{Lower bound exploiting distribution of degrees}
\label{alg:dreg}
\end{algorithm}

The input is assumed to be correct in the sense that a graph with the given degree sequence exists, and by definition, the degree sequence is ordered non-increasingly.
For each iteration $k$, Alg.~\ref{alg:dreg} stores degrees of nodes in $T^k$ in variables $a_1,\dots,a_n$.
Note that the forest $T^k$ is not actually constructed. 
The algorithm operates merely with potential degrees of nodes in $T^k$.
Next, variable $c$ keeps the value $|V_k|$, i.e., the number nodes informed within $k$ steps.
Finally, $c_n$ stores $|V_k\setminus V_{k-1}|$, thus the number of nodes newly informed in time step $k$.

\begin{proposition}
If $G$ is an arbitrary graph with $n$ nodes, $m$ sources and node degrees $d_i$, $1\leq i\leq n$, Alg.~\ref{alg:dreg} calculates a lower bound on the delay in $G$.
\end{proposition}
\begin{proof}
The assumption that nodes with higher degree in $G$ are informed first in the optimal solution is justified by Lemma \ref{lemma:degorder}.
In each iteration $k$, degrees in $T^k$ of maximum possible number of nodes are increased in the order of their decreasing degree.
The current maximum possible number of informed nodes is stored in the variable $c$, and is updated at the end of the outre loop.
So, when the condition $c<n$ is tested on line 7, $c$ contains the correct value of number of nodes in $V_k$.
As soon as it is possible that at least $n$ nodes are informed, the algorithm returns the number of necessary iterations.
\qed
\end{proof}

\section{Upper bounds}

%The algorithms presented in this section iteratively construct broadcast foerst $T_i, i=1,2,\dots$.
The following inexact method is based on the idea of finding maximum cardinality matching in $G\left[V_{T_i},V\setminus V_{T_i}\right]$, and extending $T_i$ by this matching.
It means that in each iteration, maximum possible number of nodes are informed.
Maximum cardinality matching can be regarded as finding $|V_{T_i}|$ node-disjoint  binomial trees of order at most one with roots in $V_{T_i}$, maximizing the number of edges.
By generalizing this notion, we iteratively look for $|V_{T_i}|$ node-disjoint (pruned) binomial trees of an arbitrary given order $k$ valued between 1 and $n-|S|$.
Even though this problem is NP-hard for $k\geq 2$ \cite{jansen95}, it is expected that the computational time is sound in most practical instances.
After obtaining the set of binomial trees, first $p$ nodes in each tree are selected and added to the broadcast forest.
The parameter $p\in \{1,\dots,2^k\}$ is thus a part of the input.
\begin{algorithm}[]
\KwData{$G=(V,E), S\subseteq V, k\in \{1,\dots,n-|S|\}, p\in \{1,\dots,2^k\}$}
\textbf{for }$s\in S\textbf{ do }T_s=(V_s,A_s), V_s\leftarrow \{s\}, E_s\leftarrow\emptyset$\;
$I=\{1,\dots,2^k\}$\\
$T=\{T_s:s\in S\}$\ ~~~\tcp{broadcast forest}
$t\leftarrow 0$\;
\While{$V_T\neq V$} {
	$S\leftarrow V_T$\;
	Find a set of pruned binomial trees $B=\{B_1,\dots,B_{|V_T|}\}$ of order at most $k$ with roots in $V_T$ by solving model \ref{mod:genmatch}\;
	$V_T\leftarrow V_T\cup \{v:v\in V_B:\beta(v)\leq p\}$\;
	$E_T\leftarrow E_T\cup \{\{u,v\}\in E_B: \beta(u)\leq p,\beta(v)\leq p\}$\;
	$t\leftarrow t+1$\;
}
\Return t\;
%\Return $\lceil k/s \rceil$\;
 \caption{A method for determining an upper bound}
\label{alg:match}
\end{algorithm}

Alg. \ref{alg:match} describes the process formally.
Initially, the broadcast forest consists of isolated sources.
The binomial trees are determined by solving the ILP model \eqref{mod:genmatch}. 
This model is a modification of formulation \eqref{mod:partition}, and uses the same type of variables.
The objective function is to maximize the number of nodes involved in the binomial trees.
%The index set $I=\{1,\dots,2^k\}$ depents on the input parameter $k$.
The constraints \eqref{mod:genmatch:nodeBelongs} state that each node belongs to at most one binomial tree.
Compared to \eqref{mod:part:nodeBelongs}, \eqref{mod:genmatch:nodeBelongs} is an inequality, because the binomial trees do not necessarily form a partition of $G$, and so some nodes are not used.
The remaining constraints are taken from formulation \eqref{mod:partition}. 
It is important to note that in every iteration, the model is solved with a different set $S$, as all nodes already included in the broadcast forest are roots of the binomial trees.
The model considers the entire set $V$, but it is also possible to restrict this set to nodes with distance at most $k$ from some node in $V_T$.
When this restriction is not imposed, nodes with larger distance are not a part of any binomial tree in the current iteration due to \eqref{mod:part:followArcsA} - \eqref{mod:part:followArcsB}.

\begin{subequations}\label{mod:genmatch}
\begin{align}
\notag \max\sum\limits_{v\in V}&\sum\limits_{i\in I}\sum\limits_{s\in S}   y_{is}^v,\\
\notag \text{s. t. } \\
\label{mod:genmatch:nodeBelongs} \sum\limits_{i\in I}\sum\limits_{s\in S}y^v_{is} & \leq 1 & v\in V,\\
\notag\eqref{mod:part:treeHasIJ} - \eqref{mod:part:dim}.
%\label{mod:genmatch:treeHasIJ} \sum\limits_{v\in V'}y^v_{is} & = 1 & i\in I,s\in V_T,\\
%\label{mod:genmatch:source1} y_{1s}^s & = 1  & s\in V_T,\\
%\label{mod:part:noReturn} y^u_{ij}+y^v_{lj} &\leq 1 & i\in I,l\in C(i), j\in J, u\in V_\alpha,v\in V,\\
%\label{mod:part:followArcs} y^u_{is}+y^v_{\ell s} &\leq 1 & i\in I,\ell\in C(i), s\in S, u,v\in V',(u,v)\not\in A',\\
%\label{mod:part:followArcsA} y^u_{is}+y^v_{\ell s} &\leq 1 & i\in I,\ell\in C(i), s\in S, u,v\in V,\{u,v\}\not\in E,\\
%\label{mod:part:followArcsB} y^{v_0}_{is}+y^v_{\ell s} &\leq 1 & i\in I,\ell\in C(i), s\in S, v\in V,\\
%\label{mod:genmatch:followArcsA}y^{v_0}_{is}+y^u_{i s} + \sum\limits_{v\in V\setminus N(u)}y^v_{\ell s}&\leq 1 & u\in V,i\in I,\ell\in C(i), s\in V_T,  \\
%\label{mod:genmatch:followArcsB}y^{v_0}_{is}+y^u_{\ell s} + \sum\limits_{v\in V\setminus N(u)}y^v_{i s}&\leq 1 & u\in V,i\in I,\ell\in C(i), s\in V_T,\\ 
%\label{mod:genmatch:dim}&&y \in \{0,1\}^{I\times V_T\times V'}.
\end{align}~
\end{subequations}



%
%We present a simple general method for determining an upper bound on the optimal solution.
%The idea is to construct broadcast trees in parallel from their roots in sources.
%A new node is connected to a tree is it was not connected to any other tree in any of the previous iterations.
%The steps are summarized in Alg. \ref{alg:genub}.
%
%\begin{algorithm}[]
%\KwData{$G=(V,E), S\subseteq V$}
%$U\leftarrow S$;\tcp{Set of marked nodes}
%$T_s=(V_s,A_s), V_s\leftarrow \{s\}, E_s\leftarrow\emptyset$\;
%\While{$|U|<n$} {
%	\For{$s\in S$} {
%		Select $v\in V\setminus U:\exists u\in V_s \text{ s. t.} \{u,v\}\in E$ according to some strategy\; 	
%		$U\leftarrow U\cup\{v\}$\;
%		$V_s\leftarrow V_s\cup\{v\}$\;
%		$A_s\leftarrow A_s\cup\{u,v\}$\;
%	}
%}
%\For{$s\in S$} {
%$	
%\sigma(v)=\begin{cases}
%		0, \text{ if } v\in L(T_s),\\
%		\max\limits_{k\in 1,\dots,|N^+(v)|}\{k+\sigma(j_k):j_k\in N^+(v)\wedge \forall \ell<k:\sigma(j_\ell)>\sigma(j_k)\},\\
%		\hfill \text{ if }v\in V_s\setminus L(T_s).
%
%\end{cases}\;
%$
%}
%\Return $\max_{s\in S}\sigma(s)$\;
%%\Return $\lceil k/s \rceil$\;
% \caption{A method for determining an upper bound}
%\label{alg:genub}
%\end{algorithm}
%The algorithm consists of two phases. 
%In the first phase (lines 3 -- 10), $m$ broadcast trees are constructed.
%The second phase (lines 11 -- 13) uses a recursive formula that determines in linear time an optimal broadcast time for a subtree of $T_s$ rooted at $v\in V_s$.
%The value  $\sigma(s)$ than gives the optimal broadcast time for $T_s$.
%The complexity of Alg. \ref{alg:genub} depends on the strategy according to which the trees are constructed.
%For common strategies such as BFS and DFS it is $\mathcal{O}(n+|E|)$, for the shortest paths trees $\mathcal{O}(|E| + n\log n)$.
%
%


%\begin{acknowledgements}
%If you'd like to thank anyone, place your comments here
%and remove the percent signs.
%\end{acknowledgements}

% BibTeX users please use one of
%\bibliographystyle{spbasic}      % basic style, author-year citations
%\bibliographystyle{spmpsci}      % mathematics and physical sciences
%\bibliographystyle{spphys}       % APS-like style for physics
%\bibliography{}   % name your BibTeX data base

% Non-BibTeX users please use
\begin{thebibliography}{}
%
% and use \bibitem to create references. Consult the Instructions
% for authors for reference list style.
%
\bibitem{chu17}
Chu, X., Chen, Y.,
Time division inter‐satellite link topology generation problem: Modeling and solution,
International Journal of Satellite Communications and Networking, 194 -- 206, 36 (2017)

\bibitem{cormen90}
Cormen, T. H., Leiserson, C. E., Rivest, R. L,
Introduction to Algorithms, 
MIT Press, 401 -- 402, 1990. 

\bibitem{elkin03}
Elkin, M., Kortsarz, G.,
Sublogarithmic approximation for telephone multicast: path out of jungle,
Symposium on Discrete Algorithms, 76 -- 85 (2003)

\bibitem{farley81}
Farley, A. M., Proskurowski, A.,
Broadcasting in Trees with Multiple Originators,
SIAM Journal on Algebraic Discrete Methods, 381 -- 386, 2, 4 (1981)

\bibitem{grigni91}
Grigni, M., Peleg, D.,
Tight bounds on minimum broadcast networks
Networks, 207-222, 4 (1991)

\bibitem{hasson04} 
Hasson, Y., Sipper, M.,
A Novel Ant Algorithm for Solving the Minimum Broadcast Time Problem,
International Conference on Parallel Problem Solving from Nature, 775 -- 780 (2004)

\bibitem{harutyunyan06}
Harutyunyan, H. A., Shao, B.,
An efficient heuristic for broadcasting in networks,
Journal of Parallel and Distributed Computing, 68 -- 76, 66, 1 (2006)

\bibitem{harutyunyan14}
Harutyunyan, H. A., Jimborean, C.,
New Heuristic for Message Broadcasting in Network,
IEEE 28th International Conference on Advanced Information Networking and Application, 517 -- 524, (2014)

\bibitem{jansen95}
Jansen, K., M\"uller, H.,
The minimum broadcast time problem for several processor networks, 
Theoretical Computer Science, 69 -- 85, 147 (1995)

\bibitem{kortsarz95}
Kortsarz, G., Peleg, D.,
Approximation algorithms for minimum-time broadcast
SIAM Journal on Discrete Mathematics, 401 -- 427, 8, 3 (1995)

\bibitem{mcgarvey16}
McGarvey, R. G., Rieksts, B. Q., Ventura, J. A., Ahn, N.,
Binary linear programming models for robust broadcasting in communication networks,
Discrete Applied Mathematics, 173 -- 84, 204, (2016)

\bibitem{middendorf93}
Middendorf, M.,
Minimum broadcast time is NP-complete for 3-regular planar graphs and deadline 2,
Information Processing Letters, 281 -- 287, 46 (1993)

\bibitem{noe05}
Noe, T. D., Post, J. V., 
Primes in Fibonacci n-step and Lucas n-Step Sequences,
J. Integer Seq. 8, Article 05.4.4, 2005.

\bibitem{scheuermann84}
Scheuermann, P., Wu, G.,
Heuristic Algorithms for Broadcasting in Point-to-Point Computer Networks,
IEEE Transactions on Computers, 804 -- 811, 33, 9 (1984)

\bibitem{slater81}
Slater, P. J., Cockayne, E. J., Hedetniemi, S.T.,
Information dissemination in Trees,
SIAM Journal on Computing, 692 -- 701, 10, 4 (1981)

\bibitem{wang10}
Wang, W.,
Heuristics for Message Broadcasting in Arbitrary Networks,
Master thesis, Concordia University, Montr\'eal, Qu\'ebec, 
Retrieved from http://citeseerx.ist.psu.edu/viewdoc/download?doi=10.1.1.633.5827\&rep=rep1\&type=pdf (2010)

\bibitem{jimborean13}
Jimborean, C.,
New Heuristics for Message Broadcasting in Arbitrary Networks,
Master thesis, Concordia University, Montr\'eal, Qu\'ebec, 
Retrieved from https://spectrum.library.concordia.ca/977717/1/Jimborean\_MCompSc\_F2013.pdf (2013)


% Format for Journal Reference
%Author, Article title, Journal, Volume, page numbers (year)
% Format for books
%\bibitem{RefB}
%Author, Book title, page numbers. Publisher, place (year)
% etc
\end{thebibliography}

\end{document}
% end of file template.tex

