% This is a general template file for the LaTeX package SVJour3
% for Springer journals.          Springer Heidelberg 2010/09/16
%
% Copy it to a new file with a new name and use it as the basis
% for your article. Delete % signs as needed.
%
% This template includes a few options for different layouts and
% content for various journals. Please consult a previous issue of
% your journal as needed.
%
%%%%%%%%%%%%%%%%%%%%%%%%%%%%%%%%%%%%%%%%%%%%%%%%%%%%%%%%%%%%%%%%%%%
%
% First comes an example EPS file -- just ignore it and
% proceed on the \documentclass line
% your LaTeX will extract the file if required
\begin{filecontents*}{example.eps}
%!PS-Adobe-3.0 EPSF-3.0
%%BoundingBox: 19 19 221 221
%%CreationDate: Mon Sep 29 1997
%%Creator: programmed by hand (JK)
%%EndComments
gsave
newpath
  20 20 moveto
  20 220 lineto
  220 220 lineto
  220 20 lineto
closepath
2 setlinewidth
gsave
  .4 setgray fill
grestore
stroke
grestore
\end{filecontents*}
%
\RequirePackage{fix-cm}
%
%\documentclass{svjour3}                     % onecolumn (standard format)
%\documentclass[smallcondensed]{svjour3}     % onecolumn (ditto)
\documentclass[smallextended]{svjour3}       % onecolumn (second format)
%\documentclass[twocolumn]{svjour3}          % twocolumn
%
\smartqed  % flush right qed marks, e.g. at end of proof
%
\usepackage{graphicx}
\usepackage{tkz-graph}
\usepackage{amsmath, amssymb}
\usepackage{graphicx}
\usepackage{tikz}
\usepackage{amsfonts}
\usepackage{subcaption}
\usepackage{hyperref}
%\usepackage{pdflscape} % Landscape pages
\usepackage{color}	% Different font colors
\usepackage{enumerate}
\usepackage{enumitem}
\usepackage{mathtools}
\usepackage{color}

\captionsetup{compatibility=false}

\newtheorem{observation}[theorem]{\textbf{Observation}}

\tikzset{
every edge/.style={fill=none} 
every node/.style={shape=circle,fill=gray!40,draw }}



%\newtheorem{proposition}{Proposition}
%\newtheorem{obsservation}{Observation}
%\newtheorem{corollary}{Corollary}
%newtheorem{lemma}{Lemma}
%\newtheorem{problem}{Problem}
%
% \usepackage{mathptmx}      % use Times fonts if available on your TeX system
%
% insert here the call for the packages your document requires
%\usepackage{latexsym}
% etc.
%
% please place your own definitions here and don't use \def but
% \newcommand{}{}
%
% Insert the name of "your journal" with
% \journalname{myjournal}
%
\begin{document}

\title{Insert your title here%\thanks{Grants or other notes
%about the article that should go on the front page should be
%placed here. General acknowledgments should be placed at the end of the article.}
}
\subtitle{Do you have a subtitle?\\ If so, write it here}

%\titlerunning{Short form of title}        % if too long for running head

\author{First Author         \and
        Second Author %etc.
}

%\authorrunning{Short form of author list} % if too long for running head

\institute{F. Author \at
              first address \\
              Tel.: +123-45-678910\\
              Fax: +123-45-678910\\
              \email{fauthor@example.com}           %  \\
%             \emph{Present address:} of F. Author  %  if needed
           \and
           S. Author \at
              second address
}

\date{Received: date / Accepted: date}
% The correct dates will be entered by the editor


\maketitle

\begin{abstract}
\keywords{}
% \PACS{PACS code1 \and PACS code2 \and more}
% \subclass{MSC code1 \and MSC code2 \and more}
\end{abstract}

\section{Introduction}
\label{intro}
The minimum broadcast time consists of a set of communication nodes with a subset of source nodes. 
The task is to disseminate a signal to every node in a shortest possible time while abiding by communication rules.
An \emph{informed} node is a node that has received the signal.
Otherwise, a node is \emph{non-informed}.
At the beginning, the set of informed nodes is exactly the set of sources.
An informed node $u$ can send the signal to a non-informed node $v$ if $u$ and $v$ are located within a communication vicinity of each other.

The continuous time is divided into discrete time steps.
At each time step, every informed node can forward the signal to at most one non-informed neighbor.
The number of informed nodes at some time step can therefore be up to double the number of informed nodes at the previous time step.
%\subsection{Motivation and Related Work}
%This communication protocol differs from various wireless communication models where a signal can be relayed to all nodes within a visibility range of a sender.
%An example of a practical application is a communication among processors.
%The applications are however not confined to wired networks.
%Situations where the signals have to cover large distances typically assume sending the signal to one neighbor at a time.
%This is common in satellite communication.
%
\section{Network Model and Notation}

The communication network is represented by an undirected unweighted graph $G=(V,E)$ and a subset of nodes $S\subseteq V$. 
Broadcasting is defined as a sequence of sets $S=V_0\subseteq\dots\subseteq V_k = V$ where each $V_i$ represents the nodes informed after time step $i$, $0\leq i\leq k$.
For each node $v\in V_i\setminus V_{i-1}$, there exists a single node $p(v)\in V_{i-1}$ adjacent to $v$, which forwarded the signal to $v$.
Also, for every two distinct nodes $u,v\in V_i\setminus V_{i-1}$ we have $p(u)\neq p(v)$.
The optimization problem in question is defined as follows:
\begin{problem}\label{prob:min}
Given $G=(V,E)$ and $S\subseteq V$, find a shortest sequence $S=V_0\subseteq\dots\subseteq V_k=V$ and a mapping $p:V_i\to V_{i-1}$, such that for each $v\in V\setminus S:\{v,p(v)\}\in E$, and for each $\{u,v\}\subseteq V_i\setminus V_{i-1}: p(u)\neq p(v)$.
\end{problem}
For convenience, we define the set $A=\{(i,j),(j,i):\{i,j\}\in E\}$ that consists of all arcs that can be derived by directing edges in $E$.
The set of neighbors of $v\in V$ in $G$ is denoted as $N(v)$.
\section{Exact methods}

Problem \ref{prob:min} can be formulated as an integer linear program and solved to optimality. In this section, we present two different modeling approaches. 

\subsection{Broadcast time model}
The first studied model is a straightforward formulation of the problem. Consider variables $$ x_{ij}^t=\begin{cases} 1, \text{ if } j\in V\setminus S \text{ becomes informed in time } t \text{ and } p(j)=i,\\ 0, \text{ otherwise},
\end{cases}
$$
and a variable $c$ representing the number of necessary time steps.
The worst case scenario is when $G$ is a path $v_1,\dots,v_n$ with $S=\{v_1\}$. 
In such an instance, the necessary number of time steps is $n-1$, which gives a trivial upper bound on $c$.
Problem \ref{prob:min} is then formulated as follows: 

\begin{subequations}\label{mod:basic}
\begin{align}
\label{mod:basic:obj} \min c \\ 
\label{mod:basic:onefromroot} \sum_{i \in N(s)}x^1_{si} & \leq 1 & s\in S,\\
\label{mod:basic:singlein} \sum\limits_{t=1}^{n-1}\sum\limits_{j\in N(i)}x_{ji}^t & = 1 & i\in V \setminus S,\\
\label{mod:basic:uniqueTout} \sum\limits_{j\in N(i)}x_{ij}^t & \leq 1  & t=1,\dots,n-1,i\in V,\\
\label{mod:basic:tIncreases} \sum\limits_{j\in N(i)}x_{ij}^t &\leq\sum\limits_{u=1}^{t-1}\sum\limits_{k\in N(i)\setminus\{j\}} x_{ki}^u  & i\in V\setminus S, t=2,\dots,n-1,\\
\label{mod:basic:tcrel} \sum\limits_{t=1}^{n-1}t\sum\limits_{j\in N(i)}x_{ij}^t & \leq c &  i\in V,\\
\label{mod:basic:positiveCost}x_{ij}^1 & = 0 & (i,j)\in A, i \not\in S,\\
\label{mod:basic:dim}&&x \in \{0,1\}^{A\times V}, c \in\{1,\dots,n-1\}.
\end{align}~
\end{subequations}

Constraints \eqref{mod:basic:onefromroot} indicate that source nodes transmit in the first time step.
By \eqref{mod:basic:singlein}, all non-source nodes have a single predecessor.
Constraints \eqref{mod:basic:uniqueTout} enforce that each node transmits the signal to at most one neighbor at each time step.
The requirement that only informed nodes can relay a signal is modeled by \eqref{mod:basic:tIncreases}. 
The maximum time step at which any transmission takes place is captured by \eqref{mod:basic:tcrel}, and finally, \eqref{mod:basic:positiveCost} states that a node that is not a source never transmits in the first time step.

\subsection{Binomial tree model}

The following method solves Problem \ref{prob:min} by solving a sequence of decision problems:
\begin{problem}
\label{prob:dec}
Given $G=(V,E)$, $S\subseteq V$ and $k\in \mathbb{N}$, is there a sequence $S=V_0\subseteq\dots\subseteq V_k=V$ and a mapping $p:V_i\to V_{i-1}$, such that for each $v\in V\setminus S:\{v,p(v)\}\in E$ and for each $\{u,v\}\subseteq V_i\setminus V_{i-1}: p(u)\neq p(v)$?
\end{problem}
For a deadline $k$, at most $|S|\cdot 2^k$ nodes can be informed within $k$ steps. 
%Therefore, we assume $n\leq 2^k|S|.
This can be achieved when arcs along which signals initiated at sources $s\in S$ are transmitted form binomial trees $B^k$ of order $k$ rooted at $s$.
Hence, if there is a partition of $G$ into $|S|$ truncated binomial trees of order at most $k$ rooted at sources, then $(G,S,k)$ is a YES instance of Problem \ref{prob:dec}.
Finding a partition of $G$ into $|S|$ truncated binomial trees can be equivalently formulated as finding a partition of $G$ into $|S|$ (complete) binomial trees in $G''$ constructed from $G$ as follows:
Let $\alpha\coloneqq |S|\cdot 2^k-|V|$, and let $K_\alpha=(V_\alpha,E_\alpha)$ be a complete graph on $\alpha$ nodes.
Each node in $K_\alpha$ is connected to every node $v\in V$ in the original graph $G$.
Thus, $G''=(V'',E'')$ with $V''=V\cup V_\alpha$ and $E''=E\cup E_\alpha\cup \{\{u,v\}: u\in V \wedge v\in V_\alpha\}$.
The set of arcs $A''$ is constructed by creating two arcs of opposite orientation for each edge, but arcs with orientation from $V_\alpha$ to $V$ are excluded. 
Formally, $A''=A\cup\{(u,v),(v,u): \{u,v\}\in E_\alpha\}\cup\{(u,v):u\in V \wedge v\in V_\alpha\}$.

Let $I=\{1,\dots,2^k\}$.
For a directed rooted binomial tree $B^k=(V^k,A^k)$, we define a systematic numbering of nodes in $V^k$, so that a node number determines a unique position in $B^k$.
I.e., we need a bijective mapping $\beta: V^k \to I$.
A suitable mapping $\beta$ assigns values increasingly with decreasing outgoing degree. 
If there is an ambiguity, a node whose parent has a lower number is assigned a lower number.
This mapping is defined recursively as
\begin{equation}
\label{eq:beta}
\beta(v)=\begin{cases}
1,\text{ if } v\in S \text{ is a root of } B^k,\\
\beta(p(v)) + 2^{k-deg^+(v)-1}, \text{ otherwise}.
\end{cases}
\end{equation}

\begin{proposition}
\label{lem:probeq}
For an instance $(G,S,k)$ of Problem \ref{prob:dec}, all nodes in $G$ become informed within $k$ time steps iff 
there exists a partition of $G''$ into $|S|$ directed binomial trees $B^k_1,\dots,B^k_{|S|}$ of order $k$ rooted at sources in $S$,
such that for each $B^k_i=(V^k_i,A^k_i)$ we have that $A_i\cap\{(u,v):u \in V_\alpha \wedge v\in V^k_i\}=\emptyset$. 
\end{proposition}
\begin{proof}
Assume there are directed rooted binomial trees $B^k_1,\dots B^k_{|S|}$ covering $V''$. 
Nodes in $V$ are divided into a sequence of subsets $S=V_0\subseteq\dots\subseteq V_k=V$ such that $V_i=\{v\in V:\beta(v)\in \{1,\dots,2^i\}\}$,
and $p(v)=u$ is a parent of $v$ in a corresponding binomial tree.
Observe that sets $V_{i+1}\setminus V_i$ divide nodes into equivalence classes according to their out-degree.
The out-degree decreases with increasing $i$.
This implies together with the fact that every arc $(u,v)\in A_i:deg^+(v)<deg^+(u)$, that the mapping $p$ satisfies the requirements specified by Problem \ref{prob:dec}.

Conversely, suppose there is a sequence of subsets and a mapping $p$ in $G$ with the desired properties.
For some node $u\in V_i$, there is at most $k-i$ nodes $u$ such that $p(u)=v$.
Truncated binomial trees covering $G$ can then be constructed simply by following the mapping $p$.
The remaining arcs that complement binomial trees are distributed along $A''\setminus A$.

\qed
\end{proof}

The ILP model based on partition into binomial trees uses only one type of variables
$$
y_{is}^v=\begin{cases}
1, \text{ if  node } v \text{ is the } i\text{-th node of the binomial tree rooted at } s\in S,\\
0, \text{ otherwise},
\end{cases}
$$
where $v\in V''$ and $i\in I $. 
With the definitions above, it is straightforward to specify constraints that enforce desired values for $y$-variables.
An obvious weakness of this approach is  that the number of nodes increases to $|V''|=\mathcal{O}(n|S|)$, and the dimension of variables is thus $\mathcal{O}(n^2|S|^2)$.
However, once a suitable partition is found, the arcs of binomial trees contained in $K_\alpha$ can be diversely shuffled while preserving the layour of binomial trees in $G$.
Instead of adding the entire complete graph $K_\alpha$, a single node $v_0$ with a loop $(v_0,v_0)$ is connected as an apex to the original $G$.
Let us denote this graph as $G'=(V',E')$, where $V'=V\cup\{v_0\}$, $E'=E\cup\{\{u,v_0\}:u\in V\}$. The arc set is then analogously defined as $A'=A\cup\{(u,v_0): u\in V\}$.
The requirement for partition into binomial trees has to be adjusted accordingly.
The subtrees contained in $G$ remain unchanged, every arc $(u,v)\in A_i, i=1,\dots,|S|$ in $G''$ with $u\in V$ and $v\in V_\alpha$ becomes $(u,v_0)$ in $G'$,
and every $(u,v)\in A_\alpha$ becomes $(v_0,v_0)$.
So, $v_0$ acts as a universal node that can substitute several nodes in each binomial tree.

Let us define the set $C(i)$ of $\beta$-values of children of node $v$ with $\beta(v)=i$ in $B^k$:
\begin{equation}
C(i)=\{2^l+i:l=\lceil\log_2 i\rceil,\dots,k-1\}.
\end{equation}
The values of the variables must fulfill constraints

\begin{subequations}\label{mod:partition}
\begin{align}
\label{mod:part:nodeBelongs} \sum\limits_{i\in I}\sum\limits_{j\in J}y^v_{ij} & = 1 & v\in V,\\
\label{mod:part:treeHasIJ} \sum\limits_{v\in V'}y^v_{ij} & = 1 & i\in I,j\in J,\\
\label{mod:part:source1} y_{1s}^s & = 1  & s\in S,\\
%\label{mod:part:noReturn} y^u_{ij}+y^v_{lj} &\leq 1 & i\in I,l\in C(i), j\in J, u\in V_\alpha,v\in V,\\
\label{mod:part:followArcs} y^u_{ij}+y^v_{lj} &\leq 1 & i\in I,l\in C(i), j\in S, u,v\in V',(u,v)\not\in A',\\
\label{mod:part:dim}&&y \in \{0,1\}^{I\times S\times V'}.
\end{align}~
\end{subequations}

As the model solves the decision problem, it suffices to find any feasible solution, and no objective function is needed.
The interpretation of constraints \eqref{mod:part:nodeBelongs} is that every node in the extended graph $G'$ belongs to exactly one binomial tree.
By \eqref{mod:part:treeHasIJ} is ensured that there is always exactly one $i$-th node of each binomial tree.
Next, \eqref{mod:part:source1} enforce that source nodes are the first nodes in corresponding binomial trees, in accordance with definition of the mapping $\beta$.
The remaining constraints \eqref{mod:part:followArcs} guarantee that the arcs of binomial trees follow arcs in $A'$.
The definition of $A'$ also prevents arcs of the binomial trees to be oriented from $V_\alpha$ to $V$.
In other words, once the signal leaves the original graph $G$ and enters $K_\alpha$, it cannot return back to $G$.
Without this requirement, it could be possible to find a partition of $G'$ into binomial trees, even though no partition of $G$ into truncated binomial trees exists.

\subsection{Lower bounds}
In this section, we present several lower bounding techniques. 
An optimal solution is obtained by solving a sequence of decision problems with varying delay $k$. 
It is therefore useful to determine tight lower and upper bounds in order to arrive in the optimum after solving as few decision problems as possible.
Obvious bounds for a general graph instance are given by
\begin{observation}
For an instance (G,S) of Problem \ref{prob:min},
$$\lceil\log\frac{n}{s}\rceil\leq k \leq n-1.$$
\end{observation}
These bounds can be tightened after exploiting the input graph properties. 
We aim to investigate bounds for various graph classes.

%\begin{acknowledgements}
%If you'd like to thank anyone, place your comments here
%and remove the percent signs.
%\end{acknowledgements}

% BibTeX users please use one of
%\bibliographystyle{spbasic}      % basic style, author-year citations
%\bibliographystyle{spmpsci}      % mathematics and physical sciences
%\bibliographystyle{spphys}       % APS-like style for physics
%\bibliography{}   % name your BibTeX data base

% Non-BibTeX users please use
\begin{thebibliography}{}
%
% and use \bibitem to create references. Consult the Instructions
% for authors for reference list style.
%
\bibitem{chu17}
Chu, X., Chen, Y.,
Time division inter‐satellite link topology generation problem: Modeling and solution,
International Journal of Satellite Communications and Networking, 194 -- 206, 36 (2017)

\bibitem{janseni95}
Jansen, K., M\"uller, H.,
The minimum broadcast time problem for several processor networks, 
Theoretical Computer Science, 69 -- 85, 147 (1995)

\bibitem{middendorf93}
Middendorf, M.,
Minimum broadcast time is NP-complete for 3-regular planar graphs and deadline 2,
Information Processing Letters, 281 -- 287, 46 (1993)

% Format for Journal Reference
%Author, Article title, Journal, Volume, page numbers (year)
% Format for books
%\bibitem{RefB}
%Author, Book title, page numbers. Publisher, place (year)
% etc
\end{thebibliography}

\end{document}
% end of file template.tex

