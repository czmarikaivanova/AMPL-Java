% This is a general template file for the LaTeX package SVJour3
% for Springer journals.          Springer Heidelberg 2010/09/16
%
% Copy it to a new file with a new name and use it as the basis
% for your article. Delete % signs as needed.
%
% This template includes a few options for different layouts and
% content for various journals. Please consult a previous issue of
% your journal as needed.
%
%%%%%%%%%%%%%%%%%%%%%%%%%%%%%%%%%%%%%%%%%%%%%%%%%%%%%%%%%%%%%%%%%%%
%
% First comes an example EPS file -- just ignore it and
% proceed on the \documentclass line
% your LaTeX will extract the file if required
\begin{filecontents*}{example.eps}
%!PS-Adobe-3.0 EPSF-3.0
%%BoundingBox: 19 19 221 221
%%CreationDate: Mon Sep 29 1997
%%Creator: programmed by hand (JK)
%%EndComments
gsave
newpath
  20 20 moveto
  20 220 lineto
  220 220 lineto
  220 20 lineto
closepath
2 setlinewidth
gsave
  .4 setgray fill
grestore
stroke
grestore
\end{filecontents*}
%
\RequirePackage{fix-cm}
%
%\documentclass{svjour3}                     % onecolumn (standard format)
%\documentclass[smallcondensed]{svjour3}     % onecolumn (ditto)
\documentclass[smallextended]{svjour3}       % onecolumn (second format)
%\documentclass[twocolumn]{svjour3}          % twocolumn
%
\smartqed  % flush right qed marks, e.g. at end of proof
%
\usepackage{graphicx}
\usepackage{tkz-graph}
\usepackage{amsmath, amssymb}
\usepackage{graphicx}
\usepackage{tikz}
\usepackage{amsfonts}
\usepackage{subcaption}
\usepackage{hyperref}
%\usepackage{pdflscape} % Landscape pages
\usepackage{color}	% Different font colors
\usepackage{enumerate}
\usepackage{enumitem}
\usepackage{mathtools}
\usepackage{color}
\usepackage[linesnumbered]{algorithm2e}
\SetKwRepeat{Do}{do}{while}

\captionsetup{compatibility=false}

\newtheorem{observation}[theorem]{\textbf{Observation}}

\tikzset{
every edge/.style={fill=none} 
every node/.style={shape=circle,fill=gray!40,draw }}



%\newtheorem{proposition}{Proposition}
%\newtheorem{obsservation}{Observation}
%\newtheorem{corollary}{Corollary}
%newtheorem{lemma}{Lemma}
%\newtheorem{problem}{Problem}
%
% \usepackage{mathptmx}      % use Times fonts if available on your TeX system
%
% insert here the call for the packages your document requires
%\usepackage{latexsym}
% etc.
%
% please place your own definitions here and don't use \def but
% \newcommand{}{}
%
% Insert the name of "your journal" with
% \journalname{myjournal}
%
\begin{document}

\title{Insert your title here%\thanks{Grants or other notes
%about the article that should go on the front page should be
%placed here. General acknowledgments should be placed at the end of the article.}
}
\subtitle{Do you have a subtitle?\\ If so, write it here}

%\titlerunning{Short form of title}        % if too long for running head

\author{First Author         \and
        Second Author %etc.
}

%\authorrunning{Short form of author list} % if too long for running head

\institute{F. Author \at
              first address \\
              Tel.: +123-45-678910\\
              Fax: +123-45-678910\\
              \email{fauthor@example.com}           %  \\
%             \emph{Present address:} of F. Author  %  if needed
           \and
           S. Author \at
              second address
}

\date{Received: date / Accepted: date}
% The correct dates will be entered by the editor


\maketitle

\begin{abstract}
\keywords{}
% \PACS{PACS code1 \and PACS code2 \and more}
% \subclass{MSC code1 \and MSC code2 \and more}
\end{abstract}

\section{Introduction}
\label{intro}
The minimum broadcast time consists of a set of communication nodes with a subset of source nodes. 
The task is to disseminate a signal to every node in a shortest possible time while abiding by communication rules.
An \emph{informed} node is a node that has received the signal.
Otherwise, a node is \emph{uninformed}.
At the beginning, the set of informed nodes is exactly the set of sources.
An informed node $u$ can send the signal to an uninformed node $v$ if $u$ and $v$ are located within a communication vicinity of each other.

The continuous time is divided into discrete time steps.
At each time step, every informed node can forward the signal to at most one uninformed neighbor.
The number of informed nodes at some time step can therefore be up to double the number of informed nodes at the previous time step.
%\subsection{Motivation and Related Work}
%This communication protocol differs from various wireless communication models where a signal can be relayed to all nodes within a visibility range of a sender.
%An example of a practical application is a communication among processors.
%The applications are however not confined to wired networks.
%Situations where the signals have to cover large distances typically assume sending the signal to one neighbor at a time.
%This is common in satellite communication.
%
\section{Network Model and Notation}

The communication network is represented by a connected undirected unweighted graph $G=(V,E)$ and a subset of nodes $S\subseteq V$, where $|V|=n$ and $|S|=s$. 
Broadcasting is defined as a sequence of sets $S=V_0\subseteq\dots\subseteq V_k = V$ where each $V_i$ represents the nodes informed after time step $i$, $0\leq i\leq k$.
For each node $v\in V_i\setminus V_{i-1}$, there exists a single node $p(v)\in V_{i-1}$ adjacent to $v$, which forwarded the signal to $v$.
Also, for every two distinct nodes $u,v\in V_i\setminus V_{i-1}$ we have $p(u)\neq p(v)$.
The value $k$ is referred to as \emph{delay}.
The optimization problem in question is defined as follows \cite{jansen95,middendorf93}:
\begin{problem}\label{prob:min}
Given $G=(V,E)$ and $S\subseteq V$, find a sequence $S=V_0\subseteq\dots\subseteq V_k=V$ of minimum length $k$ 
and a mapping $p:V\setminus S\to V$, such that for each $v\in V\setminus S:\{v,p(v)\}\in E$, and for each $u,v\in V_i\setminus V_{i-1}: p(u)\neq p(v)\Leftrightarrow u\neq v$.
\end{problem}
We also define the set $A=\{(i,j),(j,i):\{i,j\}\in E\}$ that consists of all arcs that can be derived by directing edges in $E$.
A degree, in-degree and out-degree of node $v$ in $G$ are denoted by $\text{deg}_G(v)$, $\text{deg}^-_G(v)$ and $\text{deg}^+_G(v)$, respectively.
Whenever there is no danger of confusion, the subscript $G$ is omitted.
The set of neighbors of $v\in V$ in $G$ is denoted by $N(v)$.

For convenience, we consider the following definition of broadcast trees \cite{grigni91}:
For $s\in S$, a broadcast tree $T_s$ with node set $V_{T_s}$ is a time-labeled directed subgraph of $G$ describing a broadcast originated by $s$ by the following rules:
\begin{enumerate}
\item $T_s$ is rooted at $s$ with arcs directed towards the leaves.
\item Each node $v$ is labeled with an integer $t(v)$, where $t(s)=0$.
\item Whenever $v$ is a parent of $u$ in $T_s$ $t(v)<t(u)$.
\item Whenever $v$ and $u$ are siblings in $T_s$, $t(v)\neq t(u)$. 
\end{enumerate}
A set of trees $T=\{T_s:\cup_{s\in S}V_{T_s}=V\}$ forms a partition of $G$ into trees referred to as \emph{broadcast forest} (or \emph{broadcast protocol}?).
$T$ can be derived from a given sequence of sets of nodes defining broadcasting and the mapping $p$.
Given a broadcast forest $T$, the delay is determined as $k=\max_{v\in V}\{t(v)\}$.
Further, let $T^i_s$ be a subtree of $T_s$ obtained by pruning all nodes $v\in V_{T_s}$ with $t(v)>i$.
Analogously, we define $T^i=\{T^i_s:s\in S\}$. 

\section{Exact methods}

Problem \ref{prob:min} can be formulated as an integer linear program. % and solved to optimality. 
In this section, we present two different modeling approaches. 

\subsection{Broadcast time model}
The first studied model is a straightforward formulation of the problem.
Consider variables 
$$ x_{uv}^k=
\begin{cases} 
1, \text{ if } v\in V_k \text{ and } \pi(v)=u,\\ 
0, \text{ otherwise},
\end{cases}
z_{k}=\begin{cases}
1, \text{ if } k\leq\tau(G,S),\\
0, \text{ otherwise},
\end{cases}
$$
and a variable $t^*$ representing the number of necessary time steps.
The worst case scenario is when $G$ is a path $v_1,\dots,v_n$ with $S=\{v_1\}$. 
In such an instance, the necessary number of time steps is $n-1$, which gives a trivial upper bound $\bar{t}=n-s$ on the value of $t^*$.
Problem \ref{prob:min} is then formulated as follows: 
\begin{subequations}\label{mod:basic}
\begin{align}
\label{mod:basic:obj} \min \sum\limits_{k=1}^{\bar{t}}z_k \\ 
%\label{mod:basic:onefromroot} \sum_{u \in N(s)}x^1_{su} & \leq 1 & s\in S,\\
\label{mod:basic:singlein} \sum\limits_{k=1}^{\bar{t}}\sum\limits_{v\in N(u)}x_{vu}^k & = 1 & u\in V \setminus S,\\
\label{mod:basic:uniqueTout} \sum\limits_{v\in N(u)}x_{uv}^k & \leq 1  & k=1,\dots,\bar{t},u\in V,\\
%\label{mod:basic:tIncreases} \sum\limits_{v\in N(u)}x_{uv}^k &\leq\sum\limits_{\ell=1}^{k-1}\sum\limits_{w\in N(u)\setminus\{v\}} x_{wu}^{\ell}  & u\in V\setminus S, k=2,\dots,\bar{t},\\
\label{mod:basic:tIncreases} x_{uv}^k &\leq\sum\limits_{\ell=1}^{k-1}\sum\limits_{w\in N(u)\setminus\{v\}} x_{wu}^{\ell}  & \{u,v\}\in V\setminus S, k=2,\dots,\bar{t},\\
\label{mod:basic:tcrel} \sum\limits_{k=1}^{\bar{t}}k\cdot x_{uv}^k & \leq t^* &  (u,v)\in A,\\
%\label{mod:basic:tcrel} \sum\limits_{t=1}^{n-1}t\sum\limits_{j\in N(i)}x_{ij}^k & \leq c &  i\in V,\\
\label{mod:basic:positiveCost}x_{uv}^1 & = 0 & (u,v)\in A, u \in V\setminus S,\\
\label{mod:basic:dim}&&x \in \{0,1\}^{A\times V},t^*\in\{1,\dots,\bar{t}\}.
\end{align}~
\end{subequations}
%Constraints \eqref{mod:basic:onefromroot} indicate that for each source node $s$, there is at most one adjacent node $u\in V_1$ such that $\pi(u)=s$.
By \eqref{mod:basic:singlein}, for every non-source node $u$, there is exactly one node $v$ such that $\pi(u)=v$.
Constraints \eqref{mod:basic:uniqueTout} enforce that for each node $u\in V$ and each subset $V_k$, there is at most one adjacent node $v\in V_k$ with $\pi(v)=u$.
The requirement that a non-source node has a neighbor $v\in V_k$ such that $\pi(v)=u$ only if there exists a node $w\in V_{k-1}$ such that $\pi(u)=w$ is modeled by \eqref{mod:basic:tIncreases}. 
%The requirement that only informed nodes can relay a signal is modeled by \eqref{mod:basic:tIncreases}. 
%The maximum time step at which any transmission takes place is captured by \eqref{mod:basic:tcrel}, and finally, \eqref{mod:basic:positiveCost} states that a node that is not a source never transmits in the first time step.
The length of the sequence of subsets is captured by \eqref{mod:basic:tcrel}, and finally, \eqref{mod:basic:positiveCost} state that if $\pi(v)\not\in S$ for some $j\in V$, then $v\not\in V_1$.
\subsection{Binomial tree model}

The binomial tree $B^k$ of order $k$ is an ordered tree defined recursively as follow \cite{cormen90}:
\begin{itemize}
\item The binomial tree $B^0$ consists of a single node.
\item The binomial tree $B^k$ has a root with $k$ children where the $i$-th child is the root of a binomial tree of order $k-i$, $i=1,\dots,k$.
\end{itemize}
An example of $B^3$ is depicted in Fig.\ref{fig:beta}.
%If a solution to MBT consists of broadcast trees that are binomial, the number of informed nodes doubles in each time step.
For a given time step $k$, the maximum number of informed nodes within $k$ steps is $|S|2^k$.
This occurs when the solution of MBT consists of broadcast trees that are binomial.
%Any broadcast tree can be regarded as pruned binomial tree.  
%Problem \ref{prob:min} can therefore be restated as finding a partition of $G$ into $m$ pruned binomial trees 
\begin{observation}
\label{obs:btspread}
if $r\in S$ is the root of $B^k$, then $\tau(B^k,\{r\})=k$.
\end{observation}
\begin{figure}
\centering
\begin{tikzpicture}[->,scale=.7,every node/.style = {scale=.6,draw,shape=circle, align=center, fill=gray!30}, level/.style={sibling distance=2.5cm/#1,level distance=1.0cm}]]
   \node[] {1}
   	   child[] { node {2} 
	   	   child {node {4}
		  	child {node {8}} 
		   }
		   child {node {6} }
	   }
   child[] { node {3}
   	   child { node {7} }
	   }
   child[] { node {5}
	}
 ;
\end{tikzpicture}
\caption{A binomial tree with nodes labeled by their $\beta$-positions}
\label{fig:beta}
\end{figure}

\subsubsection{Binomial trees over positive integers}

Let $k\in \mathbb{N}$ and $I=\{1,\dots,2^k\}$. 
A directed binomial tree $B^k=(V_{B^k},A_{B^k})$ with arcs oriented towards the leaves has a regular structure that allows to define a systematic numbering of nodes so that a node number determines unambiguously a position in $B^k$.
That is, we need an applicable bijective function $\beta:V_{B^k}\to I$.
A suitable bijection $\beta$ assigns values from $I$ to nodes increasingly with decreasing outgoing degree.
If there is an ambiguity, a node whose parent has a lower number is assigned a lower number.
This function is defined recursively as
\begin{equation*}
\label{eq:beta}
\beta(v)=\begin{cases}
1,\text{ if } v \text{ is the root of } B^k,\\
\beta(u) + 2^{k-deg^+(v)-1}, (u,v)\in A_{B^k},\text{ otherwise}.
\end{cases}
\end{equation*}
Nodes in Fig. \ref{fig:beta} are labeled with their $\beta$-values.

Pairs of integers that are assigned to adjacent nodes in a binomial tree are defined by relation
\begin{equation*}
\label{eq:betarel}
R=\{(i,j)\in\mathbb{N}\times\mathbb{N}:j=i+2^k,k\geq\log i,k\in \mathbb{Z}\}.
\end{equation*}

We further define the function $\pi':\mathbb{N}\to\mathbb{Z}_+$ that for a given integer $i$ associated with node $v$
determines $\beta$-value of parent of $v$: 
\begin{align*}
\label{eq:piprime}
\pi'(1)&=0,& \\
\pi'(j)&=j-2^{\lceil\log j\rceil -1}, &j > 1.
\end{align*}
Using functions $\alpha$, $\beta$, $\pi'$ and the relation $R$, we introduce the notion of binomial trees over positive integers.
\begin{definition}
A pair $(I,X)$ where $I=\{1,\dots,2^k\}$, $k\in\mathbb{Z}_+$, $(\alpha(u),\beta(u))=(\alpha(v),\beta(v))\Leftrightarrow u=v$ and
$X=\{(i,j)\in R, i,j\in I\}$ is an \emph{integer binomial tree of order $k$}. 

$(I,X)$, where $I\subseteq\{1,\dots,2^k\}, k\in \mathbb{Z}_+,\pi(j)\in I$ for all $j\in I\setminus\{1\}$ and
$X=\{(i,j)\in R, i,j\in I\}$ is an \emph{integer binomial tree of order $k$ pruned at $\{1,\dots,2^k\}\setminus I$}.
\end{definition}
We now develop a method for modelling Problem \ref{prob:min} whose main principle is finding a partition of $G$ into pruned binomial trees.
For any integer $i$, let $I_i=\{2^{i-1}+1,\dots,2^i\}$.
%\begin{problem}
%\label{prob:dec}
%Given $G=(V,E)$, $S\subseteq V$ and $t\in \mathbb{N}$, is there a sequence $S=V_0\subseteq\dots\subseteq V_t=V$ 
%and a mapping $\pi:V\setminus S\to V$, such that for each $v\in V\setminus S:\{v,\pi(v)\}\in E$ and for each  $u,v\in V_i\setminus V_{i-1}: \pi(u)\neq \pi(v)\Leftrightarrow u\neq v$?
%\end{problem}
%For a delay $t$, at most $s\cdot 2^t$ nodes can be informed within $t$ steps. 
%Therefore, we assume $n\leq 2^ks.
%This can be achieved when the broadcast forest $T$ consists of binomial trees $B^t$ of order $t$ rooted at sources $s\in S$.
%Hence, if there is a partition of $G$ into $s$ pruned binomial trees of order at most $t$ rooted at sources, then $(G,S,t)$ is a YES instance of Problem \ref{prob:dec}.
%Finding a partition of $G$ into $s$ pruned binomial trees can be equivalently formulated as finding a partition of $G''$ into $s$ (complete) binomial trees, 
%where $G''$ is constructed from $G$ as follows:
%Let $\alpha\coloneqq s\cdot 2^k-|V|$, and let $K_\alpha=(V_\alpha,E_\alpha)$ be a complete graph on $\alpha$ nodes.
%Each node in $K_\alpha$ is connected to every node $v\in V$ in the original graph $G$.
%Thus, $G''=(V'',E'')$ with $V''=V\cup V_\alpha$ and $E''=E\cup E_\alpha\cup \{\{u,v\}: u\in V \wedge v\in V_\alpha\}$.
%The set of arcs $A''$ is constructed by creating two arcs of opposite orientation for each edge, but arcs with orientation from $V_\alpha$ to $V$ are excluded. 
%Formally, $A''=A\cup\{(u,v),(v,u): \{u,v\}\in E_\alpha\}\cup\{(u,v):u\in V \wedge v\in V_\alpha\}$.

%\begin{observation}\label{obs:deg}
%For each $i\in\{1,\dots,t\}$, the set $\{v\in V^t: 1\leq\beta(v)\leq2^i\}\setminus\{v\in V^t:1\leq\beta(v)\leq2^{i-1}\}$ contains nodes with out-degree $t-i$.
%\end{observation}
%\begin{observation}\label{obs:childdeg}
%Children of $v\in V^t$ with $\text{deg}^+(v)=\ell$ have out-degree $0,\dots,\ell-1$.
%\end{observation}
\begin{observation}\label{obs:eqbetalevel}
$\pi'$ is injective.
%For $i_1,i_2\in I_i$, $\pi'(i_1)=\pi'(i_2)\Leftrightarrow i_1=i_2$.
\end{observation}
\begin{proof}
We notice that $\lceil\log (2^{i-1}+1)\rceil=\dots =\lceil\log 2^i\rceil$, and thus $\pi'(2^{i-1}+1),\dots,\pi'(2^i)$ are pairwise different.\qed 
\end{proof}

\begin{proposition}\label{lem:probeq}
A graph $G=(V,E)$ and $S\subseteq V$ has $\tau(G,S)\leq k$ iff it is possible to assign integers to nodes
such that they form $m$ (pruned) integer binomial trees of order $k$ or smaller rooted at sources.
\end{proposition}
\begin{proof}
Assume nodes in $V$ can be labeled by integers from $I=\{1,\dots,2^k\}$ so that they form (pruned) integer integer binomial trees $B_\ell$ of order at most $k$, $1\leq\ell\leq m$.
For $r\in V_{B_\ell}$ with $\alpha(r)=r$ and $\beta(r)=1$, by Obs. \ref{obs:btspread}, $\tau(B_\ell,\{r\})\leq k$.
The sequence of node sets $V_0,\dots,V_k$ is constructed by setting $V_0=\{v\in V:\beta(v)=1\}$, and $V_{i}=V_{i-1}\cup\{v\in V: \beta(v)\in I_i\}$.
Moreover, $\pi(v)=u\Leftrightarrow \alpha(v)=\alpha(u)~\&~\pi'(\beta(v))=\beta(u)$.
Also, $\pi(u)=\pi(v)\Leftrightarrow u=v$ follows from Obs. \ref{obs:eqbetalevel}, because nodes in $V_i$ have $\beta$ values in $I_i$.

Conversely, suppose there is a sequence of subsets and a function $\pi$ in $G$ with the desired properties.
Nodes are associated with integers from $I$ assigned according to the following steps:
\begin{enumerate}
\item $\beta(s)=1$ for all $s\in V_0$,
\item For $v\in V_i$, set $\beta(v)=j$ such that $j\in I_i$ and $(\pi'(\beta(v)),j)\in R, 1\leq i\leq k$.  
\end{enumerate}
\qed
\end{proof}
\subsubsection{The formulation}
Consider a graph $G'=(V',E')$ constructed  by adding a universal node $v_0$ to $G$. 
The set of nodes and edges is then $V'=V\cup \{v_0\}$ and $E'=E\cup\{\{v_0,v\}:v\in V\}$.
The ILP model based on partition into binomial trees uses variables
$$
y_{is}^v=\begin{cases}
1, \text{ if } \beta(v)=i \text{ and } \alpha(v)=s,\\ 
0, \text{ otherwise},\\
\end{cases}
z_{j}=\begin{cases}
1, \text{ if } j\leq\tau(G,S),\\
0, \text{ otherwise},
\end{cases}
$$
where $v\in V'$, $i\in I$, $s\in S$ and $0\leq j\leq \bar{t}$. 
With the definition of $G'$ above, it is straightforward to specify constraints that enforce desired values for $y$-variables.
Whenever $y_{is}^{v_0}=1$, it indicates that the binomial tree $B^t_s$ is pruned at node with position $i$.
%An obvious weakness of this approach is  that the number of nodes increases to $|V''|=\mathcal{O}(ns)$, and the dimension of variables is thus $\mathcal{O}(n^2s^2)$.
%However, once a suitable partition is found, the arcs of binomial trees contained in $K_\alpha$ can be diversely shuffled while preserving the layour of binomial trees in $G$.
%Instead of adding the entire complete graph $K_\alpha$, a single node $v_0$ with a loop $(v_0,v_0)$ is connected as an apex to the original $G$.
%Let us denote this multigraph as $G'=(V',E')$, where $V'=V\cup\{v_0\}$, $E'=E\cup\{\{u,v_0\}:u\in V\}\cup\{\{v_0\}\}$. 
%The arc set is then analogously defined as $A'=A\cup\{(u,v_0): u\in V\}\cup\{(v_0,v_0)\}$.
%The requirement for partition into binomial trees has to be adjusted accordingly.
%The subtrees contained in $G$ remain unchanged, every arc $(u,v)\in A^k_i, i=1,\dots,s$ in $G''$ with $u\in V$ and $v\in V_\alpha$ becomes $(u,v_0)$ in $G'$,
%and every $(u,v)\in A_\alpha$ becomes $(v_0,v_0)$.
%So, $v_0$ acts as a universal node that can substitute several nodes in each binomial tree.
Let us define the set $C(i)$ of $\beta$-positions of children (direct descendants) of node $v$ with $\beta(v)=i$ in a binomial tree of order $k$:
\begin{equation}
\label{eq:c1}
C(i)=\{2^j+i:j=\lceil\log_2 i\rceil,\dots,k-1\}.
\end{equation}
The formulation based on binomial trees is the following:
\begin{subequations}\label{mod:partition}
\begin{align}
\notag &\min\sum\limits_{j=0}^{\bar{t}}z_j,\\
\notag \text{s. t. } \\
\label{mod:part:nodeBelongs} \sum\limits_{i\in I}\sum\limits_{s\in S}y^v_{is} & = 1 & v\in V,\\
\label{mod:part:treeHasIJ} \sum\limits_{v\in V'}y^v_{is} & = 1 & i\in I,s\in S,\\
\label{mod:part:source1} y_{1s}^s & = 1  & s\in S,\\
%\label{mod:part:noReturn} y^u_{ij}+y^v_{lj} &\leq 1 & i\in I,l\in C(i), j\in J, u\in V_\alpha,v\in V,\\
%\label{mod:part:followArcs} y^u_{is}+y^v_{\ell s} &\leq 1 & i\in I,\ell\in C(i), s\in S, u,v\in V',(u,v)\not\in A',\\
%\label{mod:part:followArcsA} y^u_{is}+y^v_{\ell s} &\leq 1 & i\in I,\ell\in C(i), s\in S, u,v\in V,\{u,v\}\not\in E,\\
%\label{mod:part:followArcsB} y^{v_0}_{is}+y^v_{\ell s} &\leq 1 & i\in I,\ell\in C(i), s\in S, v\in V,\\
\label{mod:part:followArcsA}y^{v_0}_{is}+y^u_{i s} + \sum\limits_{v\in V\setminus N(u)}y^v_{\ell s}&\leq 1 & u\in V,i\in I,\ell\in C(i), s\in S,  \\
\label{mod:part:followArcsB}y^{v_0}_{is}+y^u_{\ell s} + \sum\limits_{v\in V\setminus N(u)}y^v_{i s}&\leq 1 & u\in V,i\in I,\ell\in C(i), s\in S,\\ 
\label{mod:part:yzrel}\sum\limits_{v\in V}y^v_{is} & \leq z_{\lceil\log i\rceil} & i\in I,s\in S,\\
\label{mod:part:dim}&&y \in \{0,1\}^{I\times S\times V'}, z\in \{0,1\}^{\bar{t}}.
\end{align}~
\end{subequations}

%As the model represents the decision problem, it suffices to find any feasible solution, and no objective function is needed.
The interpretation of constraints \eqref{mod:part:nodeBelongs} is that every node in the original graph $G$ belongs to exactly one binomial tree.
Note that these constraints are quantified only over $V$ and not over $V'$.
In this way it is achieved that $v_0$ can be regarded as a part of several binomial trees.
By \eqref{mod:part:treeHasIJ} is ensured that exactly one node, possibly $v_0$, is allocated to position $i$ of each binomial tree.
By the summation over $V'$ is ensured, that pruned nodes are collectively represented by $v_0$.
Next, \eqref{mod:part:source1} enforce that source nodes are always the first nodes in corresponding binomial trees, in accordance with definition \eqref{eq:beta} of the function $\beta$.
The remaining two sets of constraints guarantee that the arcs of binomial trees follow edges in $E$.
In particular, it is enforced by \eqref{mod:part:followArcsA} that if $u$ and $v$ are not adjacent in $G$, then $v$ must not act as a child of $u$ in any binomial tree.
Constraints, \eqref{mod:part:followArcsB} forbid any node from $V$ to be a child of $v_0$ in any binomial tree. 
This reflects the obvious fact that if a tree is pruned at some node, all its descendants must also be excluded from the tree.
%The definition of $A'$ also prevents arcs of the binomial trees to be oriented from $V_\alpha$ to $V$.
%In other words, once the signal leaves the original graph $G$ and enters $v_0$, it cannot return back to $G$.
Without \eqref{mod:part:followArcsA} and \eqref{mod:part:followArcsB}, it could be possible to find a feasible solution, even when no partition of $G$ into pruned binomial trees exists.
Finally, the relation \eqref{mod:part:yzrel} between $y$ and $z$ variables follows from Obs. \ref{obs:btspread}. 
It says that whenever there is a node in a position $i$, then the delay is at least $\lceil\log i\rceil$.
%Constraints \eqref{mod:part:followArcsA} and \eqref{mod:part:followArcsB} can be replaced by stronger
%\begin{subequations}
%\begin{align}
%\label{mod:part:followArcsAStrongerA}
%y^{v_0}_{is}+y^u_{i s} + \sum\limits_{v\in V\setminus N(u)}y^v_{\ell s}&\leq 1 & u\in V,i\in I,\ell\in C(i), s\in S,  \\
%\label{mod:part:followArcsAStrongerB}
%y^{v_0}_{is}+y^u_{\ell s} + \sum\limits_{v\in V\setminus N(u)}y^v_{i s}&\leq 1 & u\in V,i\in I,\ell\in C(i), s\in S. 
%\end{align}
%\end{subequations}
%
\subsubsection{Valid inequalities}
Let $W$ be a maximal independent set in $G$.
Model \eqref{mod:partition} is strengthened by 
\begin{align}
\label{mod:part:vibasic}
y^{v_0}_{is}+ \sum\limits_{v\in W}(y^v_{is}+y^v_{\ell s})&\leq 1 & i\in I,\ell\in C(i), s\in S, 
\end{align}
which exploits the fact that no pair of nodes in $W$ is adjacent, and so there must be no two nodes with adjacent $\beta$-positions.

We now generalize this idea by using the notion of graph power $G^m=(V,E^m)$ commonly defined as a graph with the same set of nodes as $G$,
and an edge between two nodes in $G^m$ is present iff there is a path of length at most $m$ between them in $G$.
For our purposes, we use a slightly modified definition of the edge set
$$E^m=\{\{u,v\}:\text{there exists a path between $u$ and $v$ in $G$ of length $m$}\}.$$
Definition \eqref{eq:c1} can be generalized to descendants of an arbitrary distance $m$ from $v$ in $B^k$:
\begin{equation}
C^{m+1}(i)=\bigcup_{j\in C^1(i)}C^m(j).
\end{equation}
For a given $m$, let $W_m$ be a maximal independent set in $G^m$.
Further strengthening of model \eqref{mod:partition} is achieved by introducing valid inequalities
\begin{align}
\label{mod:part:vigeneral}
y^{v_0}_{is}+ \sum\limits_{v\in W_m}(y^v_{is}+y^v_{\ell s})&\leq 1 & i\in I,\ell\in C^m(i), s\in S,1\leq m\leq \Delta_G-1. 
\end{align}
Clearly, inequality \eqref{mod:part:vibasic} is included in \eqref{mod:part:vigeneral} for $m=1$.
The distance between positions $i$ and $\ell=C^m(i)$ in a binomial tree is $m$.
The maximal independent set $W_m$ contains nodes such that length of any path between any two nodes is different from $m$, 
and so there must not be two nodes in $W_m$ with positions $i$ and $\ell$ at the same time.

\subsubsection{Symmetry removal}
Another improvement of this model is achieved by a symmetry removal.
If a broadcast tree is identical to a binomial tree, we notice that nodes with labels from $C(i)$, i.e., children of some node $v$ with $\beta(v)=i$ are informed in increasing time steps.
For example in $B^3$, $C(2)=\{4,6\}$ and the corresponding nodes are informed in time step 2 and 3, respectively.
If a label $\ell\in C(i)$ corresponds to a node of a binomial tree that is pruned (if $y^{v_0}_{\ell s}=1$ for some $s\in S$), 
all labels $j\in C(i)$ such that $j>\ell$ can also be pruned. 
Thus, adding 
\begin{align}
\label{mod:part:sr}
y^{v_0}_{js}&\leq y^{v_0}_{\ell s}&i\in I,j,\ell\in C(i), s\in S
\end{align}
to the model reduces the set of feasible solutions.

%An objective function is naturally lacking in the formulation of the decision problem.
%It is nevertheless straightforward to create an ILP model for the corresponding optimization problem.
%Let $z_i=1 \Leftrightarrow t=i$ be a new variable and let $\bar{t}$ be an upper bound on the delay in $(G,S)$.
%Problem \ref{prob:min} is formulated as follows:
%\begin{subequations}
%\begin{align}
%\notag &\min\sum\limits_{j=0}^{\bar{t}}z_j,\\
%\notag \text{s. t. } \\
%\notag \eqref{mod:part:nodeBelongs} - \eqref{mod:part:source1},& \eqref{mod:part:followArcsAStrongerA} - \eqref{mod:part:followArcsAStrongerB}, \eqref{mod:part:vi}, \eqref{mod:part:sr},\\
%\notag\sum\limits_{v\in V}y^v_{is} & \leq z_{\lceil\log i\rceil} & i\in I,s\in S,\\
%\notag\label{mod:part:optdim}y &\in \{0,1\}^{I\times S\times V'}, z \in \{0,1\}^{\bar{t}}.
%\end{align}~
%\end{subequations}


\section{Lower bounds}
In this section, we study lower bounds on delay $k$ for several restrictions of input graphs.
An optimal solution is obtained by solving a sequence of decision problems with varying $k$. 
It is therefore desirable to determine tight lower and upper bounds in order to arrive in the optimum after solving as few decision problems as possible.
Obvious bounds for a general graph instance are given by
\begin{observation}
For an instance (G,S) of Problem \ref{prob:min},
$$\left\lceil\log\frac{n}{s}\right\rceil\leq k \leq n-s.$$
\end{observation}

Consider a $d$-regular graph with one source $s$.
The broadcast forest $T$ consists of a single tree $T_s$.
We investigate the number of leaves in $T_s$ and derive a lower bound on the dealy for this graph class.
For $k=1,2$, $L(T^1_s)=L(T^2_s)=2$.
For $k\geq 3$, $L(T^k_s)$ corresponds to the number of nodes with degree $1,\dots,d-1$ in $T^{k-1}_s$.
It can also be interpreted as the sum of number of leaves in $T^{k-d+1}_s,\dots,T^{k-1}_s$, which leads to the following formula
\begin{equation}
\label{eq:leafrec}
L(T^k_s)=\sum\limits_{i=k-d+1}^{d-1} L(T^i_s).
\end{equation}  
Eq. \eqref{eq:leafrec} exactly corresponds to the recursive definition of Fibonacci sequence of order $d-1$.
As each of the two base cases $L(T^l_s$ and $L(T^2_s)$ equal double the base cases of the Fibonacci sequence, the number of leaves in time step $k$ is calculated as
\begin{equation}
\label{eq:fibleaf}
L(T^k_s)=2 F^{d-1}_k.
\end{equation}  
Since each node in $H^k_s$ had been a leaf in exactly one time step, the number of nodes in $T^k_s$ can be expressed as 
\begin{equation}
\label{eq:fibcnt}
|V_k|=2\sum\limits_{i=1}^k F^{d-1}_i.
\end{equation}

\begin{proposition}
For a $d$-regular graph on $n$ and $s$ source, the lower bound on delay is 
\begin{equation}
\label{lem:lbreg1}
\left\lceil\min\{k:2\sum\limits_{i=1}^k F^{d-1}_i\geq n\}/s\right\rceil.
\end{equation}
\end{proposition}
\begin{proof}
\qed
\end{proof}

An additional knowledge of a degree sequence of $G$ can be exploited. 
Alg.~\ref{alg:dreg} calculates a lower bound on the delay when given number of nodes, sources and a degree sequence in $G$ as an input.
The algorithm iteratively updates possible node degrees in $F_i$ in each time step $i$, and records the maximum potential number of nodes in $V_i$.
For the purpose of finding lower bounds, it is assumed that each node $v\in V_i$ with $\text{deg}_{F_i}(v)<\text{deg}_G(v)$ informs a new uninformed node.
The number of iterations is then the lowest possible delay for given input.
%The main idea is that in the most optimistic case, each node $v\in V_i$ with $\text{deg}_{F_i}(v)<\text{deg}_G(v)$ informs new not yet informed node.
%For the purpose of finding lower bounds, we can assume without loss of generality that $\text{deg}_{F_i}(u)\leq\text{deg}_{F_i}(v) \Leftrightarrow \text{deg}_G(u)\geq \text{deg}_G(v)$.
Once a node $v$ reaches its maximum degree, i.e., when $\text{deg}_{F_i}=\text{deg}_G(v)$ for some $i$, $v$ does not inform any other node in the next time steps.

%For each $i=1,\dots d$, Alg. \ref{alg:dreg} keeps the  number of nodes with degree $i$.
%These values are updated iteratively using dynamic programming until the number of informed nodes reaches $n$. 

\begin{algorithm}
\KwData{$n,s,d_1,\dots,d_n\in \mathbb{N}, s\leq n,\newline 1\leq d_n\leq\dots\leq d_2\leq d_1$}
$a_1,\dots,a_{2s}\leftarrow 1$\;%\tcp{this is a comment}
\For{$i=2s+1,\dots,n$} {
$a_i\leftarrow 0$\;
}
$c\leftarrow 2s$;\tcp{every source informes a new node}
$k\leftarrow 1$\;
\While {$c<n$} {
$k\leftarrow k+1$\;
$c_n\leftarrow 0$\;
\For{$i=1,\dots,c$} {
	\If {$a_i<d_i$} {
		$a_i\leftarrow a_i + 1$\;
		$c_n\leftarrow c_n + 1$\;
		\If {$c+c_n<n$} {
			$a_{c+c_n}\leftarrow 1$;\tcp{Newly informed node} 
		}
	}
}
$c\leftarrow c + c_n$\;
}
\Return $k$\;
%\Return $\lceil k/s \rceil$\;
 \caption{Lower bound exploiting distribution of degrees}
\label{alg:dreg}
\end{algorithm}


The input is assumed to be correct in the sense that a graph with given degree sequence exists, and by definition, the degree sequence is ordered non-increasingly.
For each iteration $k$, Alg.~\ref{alg:dreg} stores degrees of nodes in $F_k$ in variables $a_1,\dots,a_n$.
Note that the forest $F_k$ is not actually constructed. 
The algorithm operates merely with potential degrees of nodes in $F_k$.
Next, variable $c$ keeps the value $|V_k|$, i.e., the number nodes informed within $k$ steps.
Finally, $c_n$ stores $|V_k\setminus V_{k-1}|$, thus the number of nodes newly informed in time step $k$.


\begin{proposition}
If $G$ is an arbitrary graph with $n$ nodes, $s$ sources and node degrees $d_i$, $1\leq i\leq n$, Alg.~\ref{alg:dreg} calculates a lower bound for the delay $k$ from Problem~\ref{prob:min} in $G$.
\end{proposition}
\begin{proof}
(Idea:)
The most optimistic scenario is when $F_k$ consists of full binomial trees.
In such a case, for $u\in V_i$ and $v\in V_j$ we have $\text{deg}_{F_k}(v)\leq\text{deg}_{F_k}(u)\Leftrightarrow i\leq j$.
We can therefore assert that in an optimal solution, $\text{deg}_G(u)\leq\text{deg}_G(v)\Leftrightarrow i\leq j$.
Whenever $\text{deg}_{F_k}(v)<\text{deg}_G(v)$, $v$ informs a new node in $V\setminus V_k$ with maximum degree in $G$, which is ensured by the decreasing order of input node degrees.
For every $v\in V$, $\text{deg}_{F_k}(v)\leq \text{deg}_G(v)$ always holds.
Further, when the condition $c<n$ is tested on line 7, $c$ contains the correct value of number of nodes in $V_k$.
The returned value is therefore a lower bound on the delay.
\qed
\end{proof}





%\begin{acknowledgements}
%If you'd like to thank anyone, place your comments here
%and remove the percent signs.
%\end{acknowledgements}

% BibTeX users please use one of
%\bibliographystyle{spbasic}      % basic style, author-year citations
%\bibliographystyle{spmpsci}      % mathematics and physical sciences
%\bibliographystyle{spphys}       % APS-like style for physics
%\bibliography{}   % name your BibTeX data base

% Non-BibTeX users please use
\begin{thebibliography}{}
%
% and use \bibitem to create references. Consult the Instructions
% for authors for reference list style.
%
\bibitem{chu17}
Chu, X., Chen, Y.,
Time division inter‐satellite link topology generation problem: Modeling and solution,
International Journal of Satellite Communications and Networking, 194 -- 206, 36 (2017)

\bibitem{grigni91}
Grigni, M., Peleg, D.,
Tight bounds on minimum broadcast networks
Networks, 207-222, 4 (1991)

\bibitem{jansen95}
Jansen, K., M\"uller, H.,
The minimum broadcast time problem for several processor networks, 
Theoretical Computer Science, 69 -- 85, 147 (1995)

\bibitem{middendorf93}
Middendorf, M.,
Minimum broadcast time is NP-complete for 3-regular planar graphs and deadline 2,
Information Processing Letters, 281 -- 287, 46 (1993)

% Format for Journal Reference
%Author, Article title, Journal, Volume, page numbers (year)
% Format for books
%\bibitem{RefB}
%Author, Book title, page numbers. Publisher, place (year)
% etc
\end{thebibliography}

\end{document}
% end of file template.tex

